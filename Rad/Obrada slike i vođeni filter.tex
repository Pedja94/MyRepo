\documentclass[a4paper,12pt,titlepage]{article}
\usepackage{amsmath}
\usepackage[serbian]{babel}
\usepackage{verbatim}
\usepackage{graphicx}
\usepackage{latexsym}
\usepackage[margin=1in]{geometry}
\usepackage{color}
\usepackage{listings}
\usepackage[utf8]{inputenc}
\usepackage[T1]{fontenc}
\usepackage{currvita}
\usepackage{hyperref} 
\usepackage{float}
\usepackage[letterspace=125]{microtype}
\usepackage{tabu}
\usepackage{titlesec}
\usepackage{setspace}
\usepackage{chngcntr}

\newtheorem{definicija}{Definicija}[section]
\newtheorem{teorema}{Teorema}[section]
\renewcommand{\contentsname}{Sadr\v zaj}
\renewcommand{\refname}{Literatura}
\renewcommand{\mod}[1]{$mod$ ${#1}$}
\newcommand{\sectionbreak}{\clearpage}

%\title{\Huge {\textbf{Obrada slike korišćenjem Guided filtra}}}
%\author{\textbf{Autor:} Predrag Nikolić \\ \textbf{Mentor:} Dejan Rančić}
%\date{\today}

\begin{document}
\begin{titlepage}
    \begin{center}
    
        \includegraphics[width=2.6cm]{img/uni.png}%
        \begin{minipage}[b]{0.7\textwidth}
            \centering
            \Large
            \textbf{UNIVERZITET U NIŠU\\ ELEKTRONSKI FAKULTET\\}       
            \large
            \textbf{Katedra za računarstvo}
        \end{minipage}%
        \includegraphics[width=2.6cm]{img/elfak.png}
        
              
      
        \vspace{5cm}
        \Large
        \textbf{Primena vođenog filtera u obradi slike}
        
        \vspace{0.6cm}
        \normalsize
        \textls{-ZAVRŠNI RAD-}     
    \end{center}
    
    \vspace{1cm}
   
    %\textbf{Zadatak:}
    
   % \vspace{0.2cm}
    
    %\setlength{\leftskip}{0.6cm}
   % \noindent
   % Istražiti tehnike filtriranja slike. Implementacija vođenog filtera i analiza njegovih osobina u različitim primenama.
    
    \vspace{10cm}
    
    \noindent \begin{tabu} to \textwidth{@{}X[l] X[r]@{}}
        \textbf{Mentor:} prof. dr Dejan Rančić &
        \textbf{Kandidat:} Predrag Nikolić 14826
    \end{tabu}
    
    \vspace{1cm}
   
  % \noindent
   % Komisija:
    
   % \vspace{0.1cm}
    
   % \noindent
    %\begin{tabu} to \textwidth{@{}X[l] X[r]@{}}  
   %      \begin{tabular}{@{}ll}
     %       1. & \underline{\hspace{6.5cm}}\\
      %      2. & \underline{\hspace{6.5cm}}\\
     %       3. & \underline{\hspace{6.5cm}}
     %   \end{tabular}  
        
   % &
   
      %  \begin{tabular}{ll@{}}
       %     Datum prijave: & \underline{\hspace{3cm}}\\
    %        Datum predaje: & \underline{\hspace{3cm}}\\
     %       Datum odbrane: & \underline{\hspace{3cm}}
    %    \end{tabular}
  
  %  \end{tabu}
    
   % \vspace{4cm}
    
     \begin{center}
        Niš, \the\year.
    \end{center}
\end{titlepage}

%%%%%%%%%%%%%%%%%%%%%%%%%%%%%%%%%%%%%%%%%%%%
%Prva strana%
%%%%%%%%%%%%%%%%%%%%%%%%%%%%%%%%%%%%%%%%%%%%
\vspace{2cm}

\noindent
\textbf{Završni rad}\\


\noindent
Primena vođenog filtera u obradi slike \\


\noindent
Guided filter in image processing\\


   \noindent
    \textbf{Zadatak:}
    
   \noindent

 %\setlength{\leftskip}{0.6cm}
 \noindent
Istražiti tehnike filtriranja slike. Upoznati se sa vođenim filterom i analizirati njegove prednosti i nedostatke u odnosu na druge. Implementirati vođeni filter i uporediti njegove mogućnosti za uglačavanje i pojačavanje detalja u odnosu na osnovne filtere. Analizirati kvalitet rezultata primene vođenog filtera na uklanjanju izmaglice sa slike. 
  
\vspace{17cm}

   \noindent
    Komisija:
    
    \vspace{0.1cm}
    
    \noindent
    \begin{tabu} to \textwidth{@{}X[l] X[r]@{}}  
         \begin{tabular}{@{}ll}
            1. & \underline{\hspace{6.5cm}}\\
            2. & \underline{\hspace{6.5cm}}\\
            3. & \underline{\hspace{6.5cm}}
        \end{tabular}  
        
    &
   
        \begin{tabular}{ll@{}}
            Datum prijave: & \underline{\hspace{3cm}}\\
            Datum predaje: & \underline{\hspace{3cm}}\\
            Datum odbrane: & \underline{\hspace{3cm}}
        \end{tabular}
  
    \end{tabu}
    
    \thispagestyle{empty}
    \newpage

%%%%%%%%%%%%%%%%%%%%%%%%%%%%%%%%%%%%%%%%%%%%
%Kraj prve strane%
%%%%%%%%%%%%%%%%%%%%%%%%%%%%%%%%%%%%%%%%%%%%

\tableofcontents

\setlength{\parskip}{\baselineskip}%
\setlength{\parindent}{15pt}%
\numberwithin{equation}{section}
\numberwithin{figure}{section}
\numberwithin{table}{section}

\thispagestyle{empty}
\newpage

\pagenumbering{arabic} 

\section{Uvod}%%%%%%%%%%%%%%%%%%%%%%%%%%

Obrada slike je proces kod kojeg se slika ili video (koji je predstavljen niz slika) obrađuju da bi se dobila izmenjena slika (video) ili skup karakteristika ili nekih bitnih parametara. Metodi obrade slike su interesantni iz ugla dve primene. Jedna je vizuelno poboljšanje slike radi bolje ljudske percepcije. Druga je obrada slike u cilju čuvanja, transmisije i reprezentacije percepcije autonomnih mašina. Ova oblast ima primenu u medicini i tehnološkim oblastima, kao što je poboljšanje slika snimljenih digitalnom kamerom, prepoznavanje lica ljudi, detektovanje objekata i obrazaca na osnovu podataka sa slike itd. 

U ovom radu je dat pregled osnovnih filtara za obradu slike. Postoje dva tipa osnovnih filtara, to su filteri za uglačavanje slike i filteri za izvlačenje ivica. Prvi tip filtara se koristi kada želimo da uklonimo šum sa slike ili da zamutimo sliku u cilju uklanjanja malih detalja prilikom ekstrakcije objekata. Cilj filtara za izvlačenje ivica je naglašavanje finih detalja na slici i označavanje linije koje predstavljaju prelaze između skupova piksela sa različitim vrednostima. Njihove osobine se proučavaju i koriste za realizaciju složenih filtara. Jedan od takvih filtara je vođeni filter. Vođeni filter kombinuje osobine ova dva tipa filtara u cilju uglačavanja slike, ali tako da vodi računa da ne uglača ivice slike. Glavna prednost ovog filtera u odnosu na filtere koji imaju istu tu osobinu, je složenost koja zavisi samo od broja piksela na slici. 

Ovaj rad se bavi analizom vođenog filtera, njegovom implementacijom i nekim primenama. U radu će biti data definicija ovog filtera i biće objašnjenje najbitnije osobine. U ovom radu će biti izloženo ponašanje ovog filtera u primenama, sa posebnim osvrtom na proces uklanjanja izmaglice sa slike. Ponašanje će biti prezentovano kroz neke primere sa raznim skupovima parametara. U radu su takođe data i poređenja između vođenog filtera i drugih. Cilj ovih istraživanja je provera mogućnosti vođenog filtera i ispitivanje njegovog ponašanja u raznim situacijama. 

U drugom poglavlju je dat uvod u oblast obrade slike. U trećem poglavlju su prezentovani osnovni filteri. Četvrto poglavlje se bavi definicijom, osobinama i realizacijom vođenog filtera. U petom poglavlju je data analiza primene vođenog filtera, sa posebnim osvrtom na uklanjanje izmaglice sa slike.

\section{Obrada slike}%%%%%%%%%%%%%%%%%%%%%%%%%%

\subsection{Uvod}%%%%%%%%%%%%%%%%%%%%%%%

Digitalna slika može da se definiše kao funkcija $f(x, y)$, gde su $x$ i $y$ koordinate ravni. $x$ i $y$ predstavljaju koordinate piksela na slici. Pikseli su elementi od kojih je sastavljena slika. Obično $x$ predstavlja vrednost u odnosu na širinu slike, a $y$ u odnosu na visinu slike. Rezolucija slike se obično obeležava kao širina puta visina, izraženo u pikselima. Vrednost funkcije $f$ je intenzitet boje piksela na poziciji $(x, y)$. Intenzitet može da bude izražen kao jedna vrednost, ako je u pitanju crno bela slika, ili preko tri vrednosti ako je slika u boji. Postoje različiti načini za predstavljanje slike u boji (Npr. RGB, CMYK, HSV itd.). Vrednosti funkcije $f$ kao i $x$ i $y$ su diskretne, jer je reč o digitalnoj slici. 
 
Potreba za korišćenjem digitalnih slika javila se još 1920-tih godina u novinarstvu. Tada je prvi put slika ~\ref{slika1921} poslana preko-okenaski putem analognog kabla. Naravno prvi problemi koji su se javili tada imali su veze sa kvalitetom slike. Prvi sistemi su mogli da prenose crno bele slike, do 15 nijansi sive. Prvi kompjuteri koji su bili dovoljno moći da izvršavaju neku značajniju obradu slika javili su se 60-ih godina. Neke od prvih primena bile su vezane za poboljšanje slika iz svemira. Zatim je obrada slika počela da se koristi u medicini i astronomiji 70-ih godina, a i dan danas je bitna u tim oblastima.

\begin{figure}[ht!]
\centering
\includegraphics[width=75mm]{img/prvaPrenesenaSlika.png}
\caption{Digitalna slika koja je kreirana na osnovu kodirane trake 1921 godine. Prva slika koje je transmitovana preko-okeanski.}
\label{slika1921}
\end{figure} 

U današnje vreme skoro da ne postoji oblast u kojoj se u nekoj meri ne koristi obrada slika. Jedna od najpoznatijih primena obrade slike je kreiranje i obrada rendgenskih slika (radiografija). Ona se najviše koristi u medicini ali postoje i druge primene. Primena postoji i za slike koje predstavljaju ultravioletnu boju. Ove slike se koriste u mikroskopiji. Postoje mnoge primene u svetu tehnologije kao što su izoštravanje slika, restauracija slika, prepoznavanje lica i objekata, prepoznavanje oblika. Takođe obrada slike se koristi u filmovima i medijima.  

Bitne komponente sistema za obradu slike su specijalizovan hardver za obradu slike, kao što su grafičke kartice, zatim kompjuter koji izvršava procese obrade slike i prikazuje rezultat. Takođe je bitan softver koji se koristi za obradu slike kao i komponente i metode za čuvanje velike količine podataka. Naravno možda i najbitniji komponenta je displej na kojem možemo da vidimo rezultat obrade slika.            

\subsection{Osnovni procesi u obradi digitalne slike}%%%%%%%%%%%%%%%%%%%%%%%

Neki od osnovnih procesa i primena u obradi digitalne slike su sledeći:
\begin{itemize}
\item \emph{Pribavljanje slike} \emph{(eng. Image acquisition)} - odnosi se na to kako je slika nastala. Slika može da bude data u digitalnom obliku, a može da se zada i u analognom. Akvizicija slike uključuje predprocesiranje kao što je skaliranje i drugo.
\item \emph{Poboljšanje slike} \emph{(eng. Image enhancement)} je najprostiji ali i najkorišćeniji postupak u obradi slika. Cilj ovog procesa je da se na slici izraze bitni aspekti radi izučavanja parametra neke slike. Primene su različite (npr. izoštravanje ivica radi uočavanja detalja).
\item \emph{Restauracija slike (eng. Image restoration)} je oblast koja se bavi poboljšanjem izgleda slike. Razlika između restauracije i poboljšanja je u tome što se tehnike za restauraciju oslanjaju na matematičke i modele verovatnoće degradacije slike, dok se poboljšanje zasniva na subjektivnom osećaju čoveka. 
\item \emph{Obrada slika u boji (eng. Color image processing)} je oblast koja se bavi obradom slike u boji i artifaktima koje slike u boji donose u odnosu na obradu crno belih slika. 
\item \emph{Kompresija (eng. Compression)} se bavi tehnikama za kompresiju slike, radi smanjenja veličine slika, što je od velikog značaja za prenos i čuvanje podataka. 
\item \emph{Morfološka obrada slike (eng. Morphological processing)} se bavi alatima za ekstrakciju komponenti slika koji su pogodni za određenu reprezentaciju podataka koje slika nosi.
\item  \emph{Segmentacija (eng. Segmentation)} se bavi podelom slike na koezistentne delove ili objekte. \emph{Reprezentacija i deskripcija (eng. Representation and description)} kao što samo ime kaže se bavi predstavljanjem slike kako na ulazu tako i na izlazu nekog procesa obrade slike. 
\item  \emph{Prepoznavanje (eng. Recognition)} je proces u kojem se objektima dodeljuju labele odnosno imena na osnovu deskriptora, koji se mogu dobiti na osnovu slike. 
\end{itemize}

Svi ovi procesi mogu da koriste bazu znanja u kojoj se čuvaju bitne činjenice za svaki od procesa prilikom obrade slika. Postupci i osobine koje se javljaju na određenoj slici mogu da se jave ponovo. pa je čuvanje podataka o obradi veoma bitno. U ovom radu ćemo se fokusirati na pribavljanje slike i poboljšanje slike.

\subsection{Pribavljanje slike}%%%%%%%%%%%%%%%%%%%%%%%

Iako je oblast digitalne obrade slika izgrađena na osnovu matematičkih i formulacija u verovatnoći, pribavljanje slike je proces koji je određen na osnovu ljudskog čula vida. Slika se u digitalnom svetu pribavlja na sličan način na koji čovek vidi, odnosno kreira sliku u mozgu. Ljudski receptor za vid je oko. Prozirni prednji delovi oka lome zrake svetlosti projektujući umanjenu i obrnutu sliku na fotosenzitivnu mrežnjaču gde se u specijalizovanim nervnim ćelijama obavlja pretvaranje slike u električne nervne impulse. Zrak se prelama u očnom sočivu, a deo oka odgovoran za prikupljanje slike je žuta mrlja, gde su nervne ćelije najgušće raspoređene. Pored žute mrlje se nalazi početak vidnog živca koji je neosetljiv na svetlo, pa se njegova projekcija u vidom polju naziva crna mrlja. Nervni impulsi kreirani u ćelijama se prenosu u mozak koji spaja slike iz oba oka i okreće obrnutu sliku čime se stvara osećaj vida u ljudskom mozgu.

Svetlost koju mi opažamo ima određeni intenzitet i boju. Ljudsko oko je sposobno da vidi samo deo elektromagnetnog spektra. Ono može da detektuje svetlost koja se u spektru nalazi između ultavioletne i infracrvene boje. Posto se svetlost tretira kao talas, boju odnosno deo spektra određuje talasna dužina. Talasna dužina se predstavlja kao odnos brzine svetlosti i frekvencije svetla $\lambda = c / \nu$. Energija koju nosi talas se računa se kao $E = h \nu$, gde je $h$ Plankova konstanta, a $\nu$ je frekvencija talasa. Ljudsko oko vidi svetlost talasne dužine između $0.4 * 10^{-6}m$ i $0.7 * 10^{-6}m$.

\begin{figure}[ht!]
\centering
\includegraphics[width=120mm]{img/spektar.png}
\caption{Elektromagnetni spektar}
\label{spektar}
\end{figure} 

Slika se u digitalnom svetu generiše kao kombinacija osvetljenja prostora i reflektivne ili apsorbovane energije iz prostora scene za koju se pravi slika. Postoje različiti tipovi senzora za akviziciju slika, ovde će biti opisan sistem sa senzorom u obliku 2-D niza (matrice). Senzor je sastavljen od elemenata koji su raspoređeni u matrici. Svaki element je zadužen za prikupljanje svetlosti iz scene. U svakom elementu se beleži osvetljenje scene i tako se formira signal slike koji se kasnije dodatno obrađuje.

Pošto svetlost dolazi iz kontinulanog prostora, vrednost signala koji nosi sliku će biti kontinualna. Da bi takav signal mogao da se obradi u računaru potrebno je da se taj signal digitalizuje. Digitalizaciju signala omogućuju semplovanje i kvantizacija. Semplovanje se odnosi na digitalizaciju koordinata piksela, odnosno određivanja mesta u prostoru. Kvantizacija se odnosi na digitalizaciju vrednosti amplitude signala. Pošto je slika skup određenog broja piksela od kojih svaki ima tačno određene koordinate u prostoru $(x, y)$, signal treba da se sampluje tako da svaki piksel bude popunjen jedinstvenom vrednošću. Vrednost amplitude signala je kontinualna, pa je kao takvu nemoguće sačuvati u računaru, zbog toga se uvodi kvantizacija gde se ta vrednost prevodi u diskretan prostor. Taj proces se naziva kvantizacija. Vrednost intenziteta piksela se obično predstavlja kao ceo broj između 0 i 255. Postupak semplovanja i kvantizacije je bitan, jer od njega zavisi kvalitet i veličina podataka koje nosi signal slike. Ako je broj semplova ili kvantizacionih nivoa mali doći će do smanjenja kvaliteta slike, tj. slika neće biti verno prikazana. Međutim ako je broj semplova ili kvantizacionih nivoa veliki broj podataka za čuvanje u memoriji će biti preveliki. U ovom postupku je bitno da se pronađe balans između ove dve stvari. U zavisnosti od potrebe će biti definisan broj semplova i kvantizacionih nivoa. Npr. nekada je bitnije da se sačuva što više slika, nego da slika bude precizna. 

Slika se u digitalnom svetu predstavlja kao $f(x, y)$, gde su $x$ i $y$ koordinate piksela na slici, a $f(x, y)$ je vrednost piksela u koordinatama $(x, y)$:

\begin{equation}\label{eq:slika}
f(x, y)
=
\begin{bmatrix}
f(0, 0) & f(0, 1) & f(0, 2) & \dots & f(0, N - 1) \\
f(1, 0) & f(0, 1) & f(1, 2) & \dots & f(1, N - 1) \\
\vdots & \vdots & \vdots & \ddots & \vdots \\
f(M - 1, 0) & f(M - 1, 1) & f(M - 1, 2) & \dots & f(M - 1, N - 1)
\end{bmatrix}
\end{equation}

gde su N i M respektivno širina i visina slike.

Broj piksela slike se računa kao $b = M * N$, ako je slika crno bela, odnosno intenzitet piksela je prikazan kao jedna vrednost. Ako je u pitanju slika u boji onda se vrednost funkcije $f(x, y)$ predstavlja kao vektor vrednosti $[i_{1} i_{2} \dots i_{k}]$, gde je $k$ broj kanala, odnosno broj boja kojima je prikazana slika (najčešće 3 ako je u pitanju slika u boji). Tada se broj piksela računa kao $b = M * N * k$.

\subsection{Bitne relacije između piksela}%%%%%%%%%%%%%%%%%%%%%%%

Piksel $p$ sa koordinatama $(x, y)$ ima horizontalne i vertikalne susede $(x + 1, y)$, $(x - 1, y)$, $(x, y + 1)$, $(x, y - 1)$. Ovo se naziva 4-suseda piksela. Ako se dodaju dijagonalni susedi $(x + 1, y + 1)$, $(x - 1, y - 1)$, $(x + 1, y - 1)$, $(x - 1, y + 1)$ onda dobijamo 8-suseda piksela. Povezanost je osobina koja se dosta koristi u obradi slike. Ako su pikseli susedi i ako imaju istu vrednost ili vrednost iz istog pod-opsega mogućih vrednosti piksela onda su oni povezani. Putanja niz piksela koji su povezani na slici.

Postoje nekoliko načina za računanje distance između piksela. Ako imamo piksele $p$, $q$ i $z$ sa koordinatama $(x, y)$, $(s, t)$ i $(v, w)$, $D$ će biti mera distance ako važi:\\

\noindent
a) $D(p, q) >= 0 (D(p, q) = 0 \text{ akko } p = q)$,\\
b) $D(p, q) = D(q, p)$,\\
c) $D(p, z) <= D(p, q) + D(q, z)$.

Euklidska distanca: 

\begin{equation}\label{eq:euclid}
D_{e}(p, q) = [ (x - s)^{2} + (y - t)^{2} ]^{1/2}.
\end{equation} 

Euklidska distanca nije baš pogodna za sliku koja ima diskretne vrednosti pa se za distancu koriste:

$D_{4}$ distanca 4-suseda:

\begin{equation}\label{eq:4ne}
D_{4}(p, q) = |x - s| + |y + t|.
\end{equation} 

\[
\begin{bmatrix}
& & 2 & & \\
& 2 & 1 & 2 & \\
2 & 1 & 0 & 1 & 2 \\
& 2 & 1 & 2 & \\
& & 2 & & 
\end{bmatrix}
\].

$D_{8}$ distanca 8-suseda:

\begin{equation}\label{eq:8ne}
D_{8}(p, q) = max(|x - s|, |y + t|).
\end{equation} 

\[
\begin{bmatrix}
2 & 2 & 2 & 2 & 2 \\
2 & 1 & 1 & 1 & 2 \\
2 & 1 & 0 & 1 & 2 \\
2 & 1 & 1 & 1 & 2 \\
2 & 2 & 2 & 2 & 2 
\end{bmatrix}
\].

\subsection{Osnovne operacije nad slikom}%%%%%%%%%%%%%%%%%%%%%%%

Glavni cilj poboljšanja slike je da se slika obradi na takav način, da krajnji rezultat zadovoljava odgovarajuću primenu. tehnika koja će da se koriti za obradu slike zavisi od konkretne primene i uslova koje rezultujuća slika treba da ispuni. Obrada slike se deli na dve kategorije: Obrada slike u prostornom domenu i obrada slike u frekvencijskom domenu. Prostorni domen se odnosi na ravan u kome se slika nalazi i metode koje izvršavaju operacije nad slikom u prostoru, dok se kod drugog tipa operacije izvršavaju nad slikom koja je predstavljena u obliku signala. U ovom radu će biti objašnjena obrada slike u prostornom domenu.

Kao što je navedeno obrada slike u prostornom domenu znači da se operacije izvršavaju nad pikselima. Ove procese možemo da označimo kao $g(x, y) = T[f(x, y)]$, gde je $f(x, y)$ početna slika, $g(x, y)$ obrađena slika, a $T$ je operator koji je definisan nad pikselima i njihovim susedima na slici $f$. Piksel $(x, y)$ i njegovi susedi prave pravougaonik, čiji je centar $(x, y)$, nad kojim se izvršava operacija $T$. Vrednost piksela $g(x, y)$ se računa kao vrednost piksela $f(x, y)$ i njegovih suseda, nad kojima se izvršavaju neke operacije. Ovaj proces se obično naziva konvolucija. Osnovni postupak obrade slike se izvršava tako što se za svaki piksel slike izvrši operacija nad tim pikselom i njegovim susedima i dobijena vrednost se dodeljuje tom pikselu. Postoje dva tipa dodele: dodela na mestu računanja kada se ista slika koristi za računanje i dodelu novih vrednosti i drugi tip dodele kod kojeg se dodela vrši na drugoj slici u piksel koji ima iste koordinate kao piksel kome se pristupa na slici koja se obrađuje. 

Ako vrednosti funkcija $f(x, y)$ i $g(x, y)$ predstavimo kao $r$ i $s$ respektivno, onda obradu slike možemo da predstavimo kao $s = T(r)$. U nastavku je da pregled nekih osnovnih transformacija nad slikom u prostornom domenu. Ove transformacije će biti predstavljene nad jedno-kanalnim slikama (slike čiji pikseli uzimaju samo jednu vrednost - nijansu sive boje). Što se tiče slika u boji postupak za ove transformacije je sličan. Transformacije se ili vrše nad svakim kanalom nezavisno, ili se slika prevodi u sliku sa jednim kanalom (postoji više načina), pa se primenjuje određena transformacija.

Najosnovnija operacija koja može da se izvede nad slikom je negativ. Svaki piksel slike dobija inverznu vrednost u odnosu na svoju. Ako se vrednosti piksela kreću u granicama $[0, L - 1]$, rezultujuća vrednost se dobija kao $s = L - 1 - r$. Ova operacija se koristi na crno belim slikama koje su dominantno crne, a belom bojom su prikazani bitni podaci na slici, jer je lakše uočiti neke detalje. Rezultat primene možemo videti na slici ~\ref{negativ}. 

\begin{figure}[ht!]
\centering
\includegraphics[width=60mm]{img/img.png}
\includegraphics[width=60mm]{img/imgNegative.png}
\caption{a) slika fotografa b) negativ slike fotografa}
\label{negativ}
\end{figure}

Nad slikom može da se primeni logaritamska operacija. $s = c*log{1+ r}$. Ova operacija se koristi za izračunavanje spektra slike, što može da bude korisno za operacije koje se izvršavaju u kompleksnom domenu. Takođe nad slikom možemo da primenimo i stepene funkcije $s = c*r^{\nu}$ ili $s = c*(r + \epsilon)^{\nu}$, gde su $c$ i $\nu$ pozitivne konstante i $\epsilon$ je konstanta. Ove operacije se koriste za gama korekciju. Jedna od najpoznatijih primena je u tv uređajima koji su koriste katodne cevi. Takođe se koristi i za magnetnu rezonancu kako bi se pojačale boje i time izrazili detalji. Na slici ~\ref{transf} vidimo rezultate primena ovih operacija.

\begin{figure}[ht!]
\centering
\includegraphics[width=60mm]{img/img.png}
\includegraphics[width=60mm]{img/imgLog.png}
\includegraphics[width=60mm]{img/imgPow1.png}
\includegraphics[width=60mm]{img/imgPow2.png}
\caption{Primena raznih transformacija: a) slika fotografa b) logaritamska transformacija c) stepena transformacija $c = 1$ $\nu = 0.6$ d) stepena transformacija $c = 1$ $\nu = 0.3$}
\label{transf}
\end{figure} 

Nasuprot prethodno navedena dva pristupa postoje i linearne transformacije. Nad slikom mogu da se izvode razne linearne funkcije. Neke od najkorišćenijih su povećavanje kontrasta ili povećanje osvetljenja. Sve ove operacije imaju isti oblik $s = T(r)$, gde je T linearna transformacija, npr. menjanje osvetljenja slike može da se izrazi kao $s = r + c$ odnosno $s = r - c$, gde je $c$ konstanta u opsegu vrednosti koje uzimaju $s$ i $r$. 

Jedna od specifičnih obrada je obrada uz pomoć histograma. Histogram digitalne slike čiji pikseli uzimaju vrednosti iz $[0, L - 1]$ je diskretna funkcija $h(r_{k}) = n_{k}$, gde je $r_{k}$ k-ta vrednost u opsegu vrednosti koje uzimaju pikseli, a $n_{k}$ je broj piksela koji imaju tu vrednost. Obično se koristi normalizovana f-ja $h(r_{k}) = n_{k}/n$, gde je n $max(n_{0}, n_{1}, ... , n_{L - 1})$, pa će sve vrednosti funkcije $h$ biti u opsegu $[0, 1]$. Histogrami mogu da se prikažu i grafički i oni nam ukazuju na to koje su nijanse zastupljene i kojoj meri na slici. Primeri su dati na slikama ~\ref{hist1}, ~\ref{hist2} i ~\ref{hist3}.

\begin{figure}[ht!]
\centering
\includegraphics[width=60mm]{img/img.png}
\includegraphics[width=60mm]{img/histImg.png}
\caption{Slika fotografa i njen histogram}
\label{hist1}
\end{figure} 

\begin{figure}[ht!]
\centering
\includegraphics[width=60mm]{img/imgLog.png}
\includegraphics[width=60mm]{img/histImgLog.png}
\caption{Slika fotografa nad kojom je primenjena logaritamska transformacija i njen histogram}
\label{hist2}
\end{figure} 

\begin{figure}[ht!]
\centering
\includegraphics[width=60mm]{img/imgPow2.png}
\includegraphics[width=60mm]{img/histImgPow2.png}
\caption{Slika fotografa nad kojom je primenjena stepena transformacija i njen histogram}
\label{hist3}
\end{figure}

Obrada slike uz pomoć histograma se vrši tako što se izračuna histogram slike, nad njim se izvrši transformacija, a zatim pikseli na slici dobiju odgovarajuće vrednosti na osnovu transformisanog histograma. Nad histogramom mogu da se izvrše razne operacije u cilju poboljšanja slike. 
Jedna od operacija koja može da se izvrši uz pomoć histograma je histogramsko izjednačavanje. Cilj ovog postupka je da napravi uniformi histogram, tj. da na slici sve nijanse budu jednako zastupljene, što možemo da vidimo na primeru slike ~\ref{histeq}. Ovo je korisna transformacija ako je na slici dominiraju iste nijanse. Ako na slici dominiraju određeni broj nijansi slika može da bude nejasna, izjednačavanjem odnosno uniformisanjem histograma dobijamo sliku koja ima zastupljene sve nijanse, što će dovesti do vizuelnog poboljšanja slike. Ovde ne zalazimo detaljno u aspekte obrade slike pomoću histograma, jer ona nije od važnosti za ovaj rad.

\begin{figure}[ht!]
\centering
\includegraphics[width=60mm]{img/imgPow2.png}
\includegraphics[width=60mm]{img/histImgPow2.png}
\includegraphics[width=60mm]{img/histEq.png}
\includegraphics[width=60mm]{img/histEqhist.png}
\caption{Primena histogramskog izjednačavanja: a) bleda slika fotografa i b) njen histogram c) slika fotografa i d) njen histogram nakon histogramskog izjednačavanja}
\label{histeq}
\end{figure}

Nad slikama možemo da primenimo i aritmetičke i logičke operacije. Logičke operacije AND, OR i NOT mogu da se primene nad slikama. AND operacija može da se koristi da bi se sa slike izvukao određeni region od značaja. Region može da se izvuče tako što će se primeniti AND operacija nad slikom i maskom. Maska je slika koja sadrži informaciju koji deo slike želimo da ostavimo. Maska se formira tako što se na mestima koje želimo da sadržimo stavljamo maksimalnu vrednost koju uzimaju pikseli. Ako je opseg nijansi $[0, 1]$ onda je to 1, na svim ostalim mestima stavlja se 0. Rezultat ove operacije biće deo slike određen maskom. Na isti način se primenjuju i OR i NOT operacije. Logičke operacije su značajne za morfološku obradu slike koja je spomenuta kao jedna od oblasti kojima se bavi obrada slike. Nad slikama se često primenjuje operacija oduzimanja. Ako želimo da vidimo da li se dve slike razlikuju možemo da ih oduzmemo jednu od druge, što će da nam ukaze na razlike, jer će mesta gde su pikseli isti biti 0 tj. crna a mesta gde ima razlike će biti predstavljena određenom nijansom. U ovu svrhu može da se koristi i deljenje. Sabiranje i množenje može da se koristi u svrhu dodavanja detalja na slici.

\section{Osnovni tipovi filtara}\label{3}%%%%%%%%%%%%%%%%%%%%%%%

\subsection{Osnovni pojmovi}%%%%%%%%%%%%%%%%%%%%%%%

Kao što je ranije spomenuto prilikom obrade slike operacije se izvršavaju nad pikselom i njegovim susedima sa jedne strane i odgovarajućom pod-slikom sa druge. Ta pod slika nosi informacije o koje se koriste za filtriranje slike. Ta pod-slika (odnosno pravougaonih piksela) se naziva kernel ili prozor. Koncept filtriranja slike je uzet iz matematike odnosno analize i matematički predstavljen taj proces je u stvari konvolucija $f * w$ gde je $f$ predstavlja sliku koja se filtrira, $w$ predstavlja kernel za filtriranje. Za svaki piksel slike koja se filtrira računa se nova vrednost uz pomoć kernela $w$ i odgovarajućeg piksela i njegovih suseda, iz slike. Ako imamo kernel oblika $3 x 3$ izračunavanje vrednosti piksela $(x, y)$ izgleda ovako:

\begin{equation}\label{eq:conv1}
\begin{split}
R = w(-1, -1)f(x - 1, y - 1) + w(-1, 0)f(x - 1, y) + \dots \\
+ w(0, 0)f(x, y) + \dots + w(1, 0)f(x + 1, y) + w(1, 1)f(x + 1, y + 1).
\end{split}
\end{equation}

\begin{equation}\label{eq:conv2}
R
=
\begin{bmatrix}
f(x - 1, y - 1) & f(x - 1, y) & f(x - 1, y + 1) \\
f(x, y - 1) & f(x , y) & f(x, y + 1) \\
f(x + 1, y - 1) & f(x + 1, y) & f(x + 1, y + 1) \\
\end{bmatrix}
.*
\begin{bmatrix}
w(- 1, - 1) & w(- 1, 0) & w(- 1, 1) \\
w(-0, - 1) & w(0, 0) & w(0, 1) \\
w(1, - 1) & w(1, 0) & w(1, 1) \\
\end{bmatrix}
.\end{equation}

Koeficijent kernela $w(0, 0)$ odgovara pikselu $f(x, y)$, što ukazuje na to da je maska centrirana oko piksela $(x, y)$. Obzirom da kernel mora da bude centriran oko jednog piksela, za kernel dimenzije $m x n$, važi da je $m = 2a + 1$ i $n = 2b + 1$, gde su $a$ i $b$ pozitivni celi brojevi. Najmanji kernel koji ima smisla je $3 x 3$ , a trivijalni slučaj predstavlja kernel $1 x 1$, koji se primenjuje samo kod osnovnih operacija. Generalno filtriranje slike $f$ dimenzija $M x N$, korišćenjem kernela $w$ veličine $m x n$ izrazom:

\begin{equation}\label{eq:conv3}
g(x, y) = \sum_{s = -a}^{a} \sum_{t = -b}^{b} w(s, t) f(x + s, y + t),
\end{equation}

gde je $a = (m - 1) / 2, b = (n - 1) / 2$. Da bi se filtrirala cela slika potrebno je da se izračuna vrednost $g(x, y)$ za svako $x \epsilon [0, M - 1]$ i za svako $y \epsilon [0, N - 1]$. 

Matrica kernela može da se obeleži i ovako za primer kernela $3 x 3$:

\begin{equation}\label{eq:conv4}
R
=
\begin{bmatrix}
z_{1} & z_{2} & z_{3} \\
z_{4} & z_{5} & z_{6} \\
z_{7} & z_{8} & z_{9} \\
\end{bmatrix}
.*
\begin{bmatrix}
w_{1} & w_{2} & w_{3} \\
w_{4} & w_{5} & w_{6} \\
w_{7} & w_{8} & w_{9} \\
\end{bmatrix}
.\end{equation}

\begin{equation}\label{eq:conv5}
R = w_{1}z_{1} + w_{2}z_{2} + \dots + w_{mn}f_{mn} = \sum_{i = 1}^{mn} w_{i}z_{i}.
\end{equation}

$w_{i}$ predstavlja vrednosti kernela, $z_{i}$ predstavlja vrednosti odgovarajućih piksela slike, a $m$ i $n$ predstavljaju dimenzije kernela.

Možemo da kažemo da se filtriranje obavlja tako što kernel prolazi kroz celu sliku. Ovaj pristup se koristi kako za linearne tako i za nelinearne filtere. Bitno je obratiti pažnju na piksele slike koji se nalaze uz ivicu slike. Ti pikseli nemaju sve susede pa oni moraju na neki načina da se dodele. To može da se reši na više različitih načina, a neki od uobičajenih su postavljanje vrednosti tih suseda na 0 ili neku drugu konstantu ili uzimanje vrednosti simetričnih piksela u odnosu na piksel za koji se vrši konvolucija. 

Dva osnovna tipa filtera su filteri za uklanjanje šuma i zamućivanje slike koji se nazivaju filteri za uglačavanje i filteri za izoštravanje slike, odnosno filteri za određivanje ivica.

\subsection{Filteri za uglačavanje (''Smoothing filers'')}\label{3.2}%%%%%%%%%%%%%%%%%%%%%%%

Ovaj tip filtera se koristi za takozvano zamućivanje \emph{(eng. blurring)} i za redukciju šuma. Blurring se koristi za uklanjanje malih detalja sa slike, u procesu ekstrakcije objekata sa slike. Takođe može da se koristi za popunjavanje prekida u linijama na slici, koje mogu da se jave zbog šuma na slici. Rezultat ovog filtera je u suštini srednja vrednost piksela koji su sadržani u okolini kernela koji se koristi za filtriranje. Zbog toga se ovi filteri nazivaju i filteri srednje vrednosti \emph{(eng. averaging filters)}. 

Ideja ovog filtriranja je da se vrednost konkretnog piksela zameni sa srednjom vrednošću piksela iz njegove okoline, koji su definisani kernelom. Rezultat ovog filtera biće slika koja ima ublažene prelaze između vrednosti susednih piksela. Dva susedna piksela koja se dosta razlikuju biće sličnija po vrednosti. Pošto se pikseli, koji predstavljaju šum, dosta razlikuju po vrednost od susednih piksela, na ovaj način će se oni uklopiti u okolinu. Loša strana ovakvog filtriranja se ogleda u tome što će mesta koja predstavljaju ivice ili linije na slici početi da gube oštrinu. Naime na mestima na kojima se nalaze ivice imamo veliku razliku u susednim pikselima, pa će primenom ovog filtera ta razlika postati manja što će dovesti do gubljenja oštrine tih ivica.

\[
1/9
*
\begin{bmatrix}
1 & 1 & 1 \\
1 & 1 & 1 \\
1 & 1 & 1 \\
\end{bmatrix}
1/16
*
\begin{bmatrix}
1 & 2 & 1 \\
2 & 4 & 2 \\
1 & 2 & 1 \\
\end{bmatrix}
\]. 
Ovo su tipični primeri filtera za uglačavanje dimenzija $3 x 3$. Levo je najprostiji primer ovog tipa filtara. Primenom ovog kernela svaki piksel će dobiti kao rezultat prosečnu vrednost susednih piksela koji odgovaraju obliku ovog kernela. Koeficijent $1/9$ služi da bi se dobila srednja vrednost. Rezultat koji će biti upisan u piksel je $R = 1/9 \sum_{i = 1}^{9} z+{i}$. Kao što vidimo na primeru kernela sa desne strane, ne moraju svi susedni pikseli da učestvuju podjednako u sumi. Na ovom primeru vidimo da pikseli različito učestvuju u sumi sa različitim koeficijentima koji su definisani u kernelu. Te vrednosti se nazivaju težine. Vidimo da se i koeficijent koji stoji uz kernel razlikuje u odnosu na primer levo. On je $1/16$, jer je suma brojeva iz kernela $16$. U većini slučajeva se ovaj koeficijent određuje na ovaj način. Na ovom primeru vidimo da se prilikom filtriranja daje prednost samom pikselu, jer uz njega stoji najveći koeficijent 4, zatim se daje prednost horizontalnim i vertikalnim susedima, oni imaju koeficijente 2 i tek na kraju su dijagonalni susedi sa koeficijentom 1. Ovi koeficijenti u nekim slučajevima mogu da budu i 0. Primer filtriranja je dat na slici ~\ref{averageFilter}.

\begin{figure}[ht!]
\centering
\includegraphics[width=60mm]{img/img.png}
\includegraphics[width=60mm]{img/imgAvg3.png}
\includegraphics[width=60mm]{img/imgAvg5.png}
\includegraphics[width=60mm]{img/imgAvg9.png}
\caption{Primena filtera srednjih vrednosti: a) slika fotografa b) slika fotografa filtrirana kernelom 3x3 c) 5x5 i d) 9x9}
\label{averageFilter}
\end{figure}

Naravno ova priča važi i za filtere koji koriste kernel bilo koje dimenzije $m x n$ ($m$ i $n$ su neparni). Ako imamo kernel ovih dimenzija opšta formula za izračunavanje vrednosti rezultujućeg piksela izgledaće ovako: 

\begin{equation}\label{eq:smoot1}
g(x, y) = \dfrac{\sum_{s = -a}^{a} \sum_{s = -b}^{b} w(s, t)f(x + s, y + t)}{\sum_{s = -a}^{a} \sum_{s = -b}^{b} w(s, t)}, 
\end{equation}

gde su $g$ i $f$ ulazna i rezultujuća slika, $(x, y)$ koordinate piksela za koji se trenutno računa vrednost, $a = (m - 1) / 2$, $b = (n - 1) / 2$, $w$ težine u kernelu. Da bi cela slika bila filtrirana ova funkcija treba da se izvrši za svako $x \epsilon [0, M - 1]$ i za svako $y \epsilon [0, N - 1]$. 

Kao što je navedeno kerneli ovih filtera su proizvoljnih veličina. Za ovaj tip filtera se obično koriste $n x n$ kerneli, jer oni uzimaju simetrične piksele i najbolje određuju koliko se neki piksel razliku je u odnosu na njegovu okolinu. Težine definisane u kernelu određuju kojim susedima se daje prednost prilikom računanja srednje vrednosti. Npr. ako neka slika ima dosta vertikalnih ivica, težina koja će stojati uz piksele, koji su vertikalni susedni u odnosu na piksel za koji se računa vrednost, može da bude 0 da bi se izbeglo gubljenje oštrine. Veličina dimenzija kernela utiče na to koliko se brojčano uzima suseda, što će na krajnjem rezultatu uticati na to u kojoj je količini slika zamućena. Što su veće dimenzije kernela, veća će biti zamućenost na slici. Pošto je svaka slika drugačija, ona će zahtevati određenu kombinaciju težina i dimenzije kernela. Ovo se obično određuje probom različitih kernela da bi se utvrdilo koji je najbolji za konkretnu sliku. 

Kao što je spomenuto ranije jedna od primena ovog tipa filtara je pred-procesiranje slike zarad uklanjanja sitnih detalja koji nisu od važnosti (slika ~\ref{zvezde}). Ovo se primenjuje prilikom ekstrakcije objekata sa slike mogu da se u određenoj meri uklone sitni objekti da bi se olakšala ekstrakcija bitnih objekata. Veličina kernela će odrediti koja koji će objekti biti uklonjeni sa slike. Ako se koristi kernel veličine $3 x 3$ sa slike će biti uklonjene samo tačke veličine jedan piksela. Povećanjem dimenzija kernela biće uklonjene i veći objekti. Odabir veličine kernela zavisiće od toga šta želimo da odstranimo sa slike.

\begin{figure}[ht!]
\centering
\includegraphics[width=60mm]{img/tel.jpg}
\includegraphics[width=60mm]{img/telAvg.jpg}
\includegraphics[width=60mm]{img/telTh.jpg}
\caption{Izdvajanje objekata sa slike a) slika ''Hubble'' teleskopa b) slika filtrirana kernelom 9x9 c) slika pretvorena u binarnu sliku(samo crna i bela)}
\label{zvezde}
\end{figure}

Pored navedenih linearnih filtera postoje i nelinearni filteri bazirani na sortiranju vrednosti piksela koji se nalaze u oblasti koja je obuhvaćena filtriranjem, odnosno određena kernelom. Najpoznatiji primer je medijan filter. Vrednosti u kernelu odnosno težine koje se koriste u ovom filteru su 1. Pikseli iz željene oblasti se sortiraju i bira se vrednost koja se u tako sortiranom nizu piksela nalazi na sredini. Ona se dodeljuje pikselu koji se trenutno obrađuje. Ako koristimo kernel veličine $3 x 3$ nakon što sortiramo njihove vrednosti za rezultat ćemo odabrati 5-ti piksel po veličini (nije bitno da li se sortiraju rastuće ili opadajuće). Dakle vrednost piksela se zamenjuje sa pikselom iz njegove okoline koji ima statistički srednju vrednost. medijan filter je efikasan za uklanjanje slučajnog šuma, a posebno je efikasan za uklanjanje šuma koji se naziva biber i so, a specifičan je po tome što se iskazuje preko crnih i belih tačaka, koje su minimalna i maksimalna vrednost koju piksel može da uzme, pa se zbog toga naziva i impulsni šum. Za ove primene medijan filter je efikasniji od prethodno navedenih filtera, jer bira vrednost koja se najviše javlja u okolini određenog piksela (slika ~\ref{medijan}).

\begin{figure}[ht!]
\centering
\includegraphics[width=60mm]{img/img.png}
\includegraphics[width=60mm]{img/imgMed.png}
\caption{Primena medijan filtera: a) slika fotografa b) slika fotografa filtrirana medijan filterom}
\label{medijan}
\end{figure}

\subsection{Filteri za određivanje ivica (''Sharpening filteri'')}%%%%%%%%%%%%%%%%%%%%%%%

Glavni cilj ovog tipa filtara je da na slici obeleže fine detalje, kao što su linije i prelazi, ili da poboljša izgled tih detalja, koji mogu biti mutni ili da imaju grešku, koja je posledica određenog metoda pribavljanja slike. Primena ovih filtara je veoma široka i varira od elektronskog štampanja preko medicine do industrijske inspekcije i autonomnih vojnih i sistema nadzora. 

Filteri opisani u pot-poglavlju ~\ref{3.2} koriste sume i srednje vrednosti da bi filtrirali sliku. Ovaj proces je analogan integraciji, pa možemo da zaključimo da filteri za određivanje ivica, koji imaju suprotni cilj u odnosu na filtre za uglačavanje, koriste diferenciranje da bi postigli željeni rezultat. Delovanje funkcije izvoda je proporcionalno stepenu nepovezanosti slike u tački, gde je izvod primenjen. Diferenciranje slika će naglasiti ivice i prelaze, kao i šum i sve delove gde se vrednosti piksela naglo menjaju (prelaz između dve boje na slici koje se dosta razlikuju u nijansi npr. crna i bela).

Da bi objasnili uticaj izvoda na sliku, objasnićemo osnovne koncepte preko jednodimenzionalnih izvoda. Izvod diskretne funkcije se definiše kao razlika između dva uzastopna elementa. Postoji više načina za definisanje izvoda nad slikom, ali svi oni zahtevaju da funkcija izvoda ispunjava sledeće:
\begin{itemize}
\item 1) jednaka je nuli u ravnim segmentima (segmenti koji imaju konstantnu vrednost) 
\item 2) različita je od nule u segmentima gde postoji nagla ili konstantna promene 
\item 3) različita je od nule duž segmenata gde se javlja konstantna promena
\end{itemize}

Slično i za funkciju drugog izvoda važi da je: 

\begin{itemize}
\item 1) jednaka je nuli u ravnim segmentima 
\item 2) različita je od nule na mestima gde se završava nagla ili konstantna promena 
\item 3) jednaka je nuli duž segmenata gde se javlja konstantna promena 
\end{itemize}

S obzirom da pikseli slike uzimaju diskretne vrednosti vrednost izvoda se kreće u granicama tih vrednosti. Definicija jednodimenzionalnog izvoda je:

\begin{equation}\label{eq:izvod}
\dfrac{\delta f}{\delta x} = f(x + 1) - f(x), \text{parcijalni izvod po x osi}.
\end{equation}

\begin{equation}\label{eq:izvod123}
\dfrac{\delta f}{\delta y} = f(y + 1) - f(y), \text{parcijalni izvod po y osi}.
\end{equation}

Slično se za drugi izvod definiše:

\begin{equation}\label{eq:izvod2}
\dfrac{\delta^2 f}{\delta x} = f(x + 1) + f(x - 1) - 2f(x), \text{drugi parcijalni izvod po x osi}.
\end{equation}

\begin{equation}\label{eq:izvod1234}
\dfrac{\delta^2 f}{\delta y} = f(y + 1) + f(y - 1) - 2f(y), \text{drugi parcijalni izvod po y osi}.
\end{equation}

\begin{figure}[ht!]
\centering
\includegraphics[width=120mm]{img/izvod.png}
\caption{Prvi i drugi izvod nad nizom piksela}
\label{overflow}
\end{figure}

Ako pogledamo sliku 10 možemo da uočimo da je prvi izvod različit od nule u segmentu gde je pad konstantan (svi susedi se razlikuju za 1), dok je drugi izvod različit od nule na početku i kraju tog segmenta. Pošto se ivice na slici iskazuju u ovom tranzicijom, možemo da zaključimo da će prvi izvod da proizvede debele ivice, dok će drugi izvod proizvesti tanke ivice. Npr. ako imamo belu liniju na crnoj pozadini, koja ima određenu debljinu, prvi izvod će izdvojiti celu liniju. Drugi izvod će ,sa druge strane, izdvojiti dve finije linije koje ukazuju na prelaz sa crne na belu boju i sa bele na crnu liniju. Ako pogledamo deo na kome imamo izolovani skok u vrednosti piksela, možemo da primetimo da je drugi izvod izraženiji nego prvi. Ovo je normalno jer je drugi izvod bolji u pronalaženju oštrih promena. Na osnovu ovoga možemo da očekujemo da će drugi izvod bolje da pronalazi finije detalje i šum. Ako je u pitanju manja promena prvi i drugi izvod će se ponašati slično. Na osnovu svega rečenog možemo da zaključimo da primenom prvog izvoda dobijamo debele ivice, dok primenom drugog izvoda dobijamo finije linije i izolirane tačke. Prvi izvod nam ukazuje na mesta gde je promena konstantna. Drugi izvod nam ukazuje na mesta gde je promena velika i izolovana. Uglavnom se drugi izvod više koriti u konkretnim primenama zbog toga što naglašava finije detalje. Zbog toga ćemo se ovde fokusirati na filtere koje koriste drugi izvod. Bitno je naglasiti da se filteri koji koriste prvi izvod analogno koriste i imaju iste osobine kada se koriste.

Nad slikom se primenjuju dvodimenzionalni izvodi, jer slika ima dve dimenzije. Na osnovu ovoga se formiraju kerneli koji se koriste u primeni ovih filtara. Nas interesuju izotropski filteri, tj. filteri koji su invarijantni u odnosu na pravac promena, koje se javljaju na slici. Drugim rečima ovi filteri su nezavisni od rotacije, što znači da će pronaći sve ivice bez obzira na njihov ugao u ravni. Ako sliku rotiramo ovakav filter će pronaći iste ivice kao na originalnoj slici. Najprostija funkcija izvoda koja se koristi je \emph{Laplacian}:

\begin{equation}\label{eq:grad}
\Delta^{2}f = \dfrac{\delta^{2}f}{\delta x^{2}} + \dfrac{\delta^{2}f}{\delta x^{2}}. 
\end{equation}

Pošto su izvodi linerane funkcije možemo da zaključimo da je i \emph{Laplacian} linearna funkcija. Da bi smo ovu funkciju mogli da koristimo za obradu slike moramo da je prevedemo u diskretnu formu. Iskoristićemo funkciju koju smo koristili da definišemo jednodimenzionalni izvod. Drugi parcijalni izvod po x-osi će izgledati ovako:

\begin{equation}\label{eq:grad2}
\dfrac{\delta^{2}f}{\delta x^{2}} = f(x + 1, y) + f(x - 1, y) - 2f(x, y).
\end{equation}

Slično, drugi parcijalni izvod po y-osi će izgledati ovako:

\begin{equation}\label{eq:grad3}
\dfrac{\delta^{2}f}{\delta y^{2}} = f(x, y + 1) + f(x, y - 1) - 2f(x, y).
\end{equation}

Na osnovu ovoga možemo da zaključimo da će, ako ova dva parcijalna izvoda zamenimo u formuli ~\ref{eq:grad}, \emph{Laplacian} funkcija izgledati ovako:

\begin{equation}\label{eq:grad4}
\Delta^{2}f = f(x + 1, y) + f(x - 1, y) + f(x, y + 1) + f(x, y - 1) - 4f(x, y). 
\end{equation}

Ovako implementirana funkcija može da se predstavi sledećim kernelom.

\[
\begin{bmatrix}
0 & 1 & 0 \\
1 & -4 & 1 \\
0 & 1 & 0 \\
\end{bmatrix}
\]. 

Da bi se u obzir uzele i promene koje se javljaju dijagonalno u odnosu na tačku koju posmatramo koristi se modifikovan kernel koji izgleda ovako.

\[
\begin{bmatrix}
1 & 1 & 1 \\
1 & -8 & 1 \\
1 & 1 & 1 \\
\end{bmatrix}
\].

Ovde se i dijagonalni susedi uzimaju obzir. Ovo je kao da imamo definisana još dva parcijalna izvoda nad dijagonalama, pa se za obe dijagonale dodaju $-2f(x, y)$. Zbog toga je koeficijent koji odgovara pikselu $(x, y)$ jednak -8. Takođe mogu da se koriste i kerneli koji imaju težine koje su obrnutog znaka u odnosu na prethodno navedene. Oni će dati iste rezultate, ali njihov znak treba da se uzme u obzir kada se ovaj pristup kombinuje sa drugim operacijama, jer će rezultati biti drugačijeg znaka.

\[
\begin{bmatrix}
0 & -1 & 0 \\
-1 & 4 & -1 \\
0 & -1 & 0 \\
\end{bmatrix}
\begin{bmatrix}
-1 & -1 & -1 \\
-1 & 8 & -1 \\
-1 & -1 & -1 \\
\end{bmatrix}
\]. 

Pošto je \emph{Laplacian} funkcija izvoda, ona će na slici otkriti promene u nijansama, kao što su tačke, ivice i linije (slika ~\ref{mesec}). Možemo da zaključimo da će rezultat primene \emph{Laplacian}-a biti slika gde će promene biti označene nekim nijansama bele i sive, a delovi u kojima nema promena biće crni. Ako želimo da na originalnoj slici naglasimo ivice mi možemo da dodamo vrednosti rezultata primene \emph{Laplacian}-a odgovarajućim pikselima. Pošto \emph{Laplacian} može da da i negativne, moramo da obratimo pažnju na to koji oblik kernela koristimo pozitivni ili negativni. Ako je središnji element kernela pozitivan onda ćemo rezultat dodati, a u suprotnom oduzeti od slike.

\begin{figure}[ht!]
\centering
\includegraphics[width=65mm]{img/moon.jpg}
\includegraphics[width=65mm]{img/moonLap.jpg}
\includegraphics[width=65mm]{img/moonEn.jpg}
\caption{Pojačavanje detalja na slici: a) slika meseca b) primena Laplacian filtera c) rezultat sabiranja alike meseca i njenog laplacian-a}
\label{mesec}
\end{figure}

Ova operacija može da se implementira u jednom prolazu tako što će da se modifikuje kernel. Operacija koja je izvršena na prethodnoj slici može da se predstavi ovako:

\begin{equation}\label{eq:grad10}
g(x, y) = f(x, y) - \Delta^{2}f(x, y) = 5f(x, y) - [f(x + 1, y) + f(x - 1, y) + f(x, y - 1) + f(x, y + 1)].
\end{equation} 

\newpage
Na osnovu funkcije ~\ref{eq:grad10} možemo da zaključimo da će kernel da izgleda ovako:

\[
\begin{bmatrix}
0 & -1 & 0 \\
-1 & 5 & -1 \\
0 & -1 & 0 \\
\end{bmatrix}
\begin{bmatrix}
-1 & -1 & -1 \\
-1 & 9 & -1 \\
-1 & -1 & -1 \\
\end{bmatrix}
\]. 

Postoji još jedan način za postizanje poboljšanja slike. Naime ivice slike mogu da se odrede tako što će od originalne slike oduzeti slika nad kojom je primenjen filter za uglačavanje. Pošto filter za uglačavanje ima suprotno dejstvo od filtera za izvlačenje ivica, oduzimanje će dati isti rezultat kao primena filtera za određivanje ivica. 

Kao što je rečeno ranije za filtriranje slike može da se koristi i prvi izvod. Prvi izvod se za filtriranje koristi tako što se računa dužina gradijenata po x i y osi. Za funkciju $f(x, y)$ gradijent u koordinatama $(x, y)$ se definiše kao vektor:

\begin{equation}\label{eq:grad11}
\Delta \textbf{f}
=
\begin{bmatrix}
G_{x} \\
G_{y} \\
\end{bmatrix}
=
\begin{bmatrix}
\dfrac{\delta f}{\delta x} \\
\\
\dfrac{\delta f}{\delta y} \\
\end{bmatrix}
\end{equation}\\

Dužina vektora se računa kao: 

\begin{equation}\label{eq:grad12}
\Delta \emph{f} = mag(\Delta \textbf{f}) = [G_{x}^{2} + G_{y}^{2}]^{1/2} = [ (\dfrac{\delta f}{\delta x})^{2} + (\dfrac{\delta f}{\delta y})^{2} ]^{1/2}. 
\end{equation}

Gradijenti su linearne funkcije, ali njihova funkcija dužine nije jer sadrži stepene operacije. Sa druge strane gradijenti nisu invarijantni u odnosu na rotaciju dok funkcija njihove dužine jeste, pa se zbog toga ona koristi. Iako nije tačno dužina vektora gradijenta se često naziva gradijent. Pošto nije lako primeniti ovu funkciju nad slikom koja ima diskretne vrednosti za potrebe obrade slike se obično koristi aproksimacija dužine gradijenta $\Delta \approx |G_{x}| + |G_{y}|$. Ova funkcija je lakša za izračunavanje i prepoznaje promene, ali generalno gubi osobinu invarijacije u odnosu na rotaciju. Kao što je slučaj kod \emph{Laplacian}-a izotropska osobina očuvana je zahvaljujući odabiru maske i dodavanju dijagonalnih elemenata. 

Ako je region na slici za kernel veličine 3 x 3 definisan kao

\[
\begin{bmatrix}
z_{1} & z_{2} & z_{3} \\
z_{4} & z_{5} & z_{6} \\
z_{7} & z_{8} & z_{9} \\
\end{bmatrix}
\].

gde $z_{5}$ predstavlja $f(x, y)$, a $z_{1}$ predstavlja $f(x - 1, y - 1)$ onda je $G_{x} = (z_{9} - z_{5})$ i $G_{y} = (z_{8} - z_{6})$. Ovo je prva predložena definicija za računanje prvih izvoda. Sledi da je gradijent:

\begin{equation}\label{eq:grad13}
\Delta \emph{f} \approx |z_{9} - z_{5}| + |z_{8} - z_{6}|.
\end{equation}

Kernel za ovako definisan gradijent se naziva Robertsov kros-gradijent operator i izgleda ovako:

\[
\begin{bmatrix}
-1 & 0 \\
0 & 1 \\
\end{bmatrix}
\begin{bmatrix}
0 & -1 \\
1 & 0 \\
\end{bmatrix}
\].

Pošto za kernele nije standardno da imaju parne dimenzije uvodi se ovakva definicija gradijenta. Ako imamo kernel veličine $3 x 3$ funkcija gradijenta izgleda ovako: 

\begin{equation}\label{eq:grad14}
\Delta \emph{f} \approx |(z_7 + 2z_8 + z_9) - (z_1 + 2z_2 + z_3)| + |(z_3 + 2z_6 + z_9) - (z_1 + 2z_4 + z_7)|.
\end{equation}

\newpage
Kerneli za ovu funkciju izgledaju ovako: 

\[
\begin{bmatrix}
-1 & -2 & -1 \\
0 & 0 & 0 \\
1 & 2 & 1 \\
\end{bmatrix}
\begin{bmatrix}
-1 & 0 & 1 \\
-2 & 0 & 2 \\
-1 & 0 & 1 \\
\end{bmatrix}
\].
.
Ovi kerneli su poznatiji kao Sobelovi operatori. Rezultat filtriranja Sobelovim operatorima je prikazan na slici ~\ref{sobel}.

\begin{figure}[ht!]
\centering
\includegraphics[width=60mm]{img/img.png}
\includegraphics[width=60mm]{img/imgSob.png}
\caption{a) slika fotografa b) primena sobel filtera nad slikom}
\label{sobel}
\end{figure}

\subsection{Kombinovanje filtara}%%%%%%%%%%%%%%%%%%%%%%%

Prethodno prikazani filteri za uklanjanje šuma i određivanje ivica imaju veliku primenu u obradi slike. U većini slučaja oni se koriste kao deo nekog većeg procesa. U prethodnim poglavljima je navedeno nekoliko slučajeva. Filteri za uglačavanje se koriste da bi se uklonili objekti koji nisu od važnosti za ekstrakciju objekata sa slike. Takođe ovi filteri mogu da se koriste za određivanje ivica i poboljšanje slike, tako što će se na sliku dodati detalji dobijeni oduzimanjem slike i njene zamućene verzije, koja se dobija primenom filtera za uglačavanje. Detalji dobijeni primenom filtara za određivanje ivica i detalja mogu da se dodaju na sliku i cilju poboljšanja i naglašavanja detalja. 

Jasno je da se ovi filteri koriste u kombinaciji sa drugim filterima i kao deo nekog većeg procesa. Oni mogu međusobno da se kombinuju. Postoje i složeni filteri koji su napravljeni koristeći koncepte ova dva tipa filtara. To je najmoćniji tip filtara koji kao rezultat daje slike kod kojih se uklanja šum, što je odlika filtara za uglačavanje, ali se pri tome slika ne zamućuje, čuvaju se ivice, što je odlika filtara za određivanje ivica. Dva najpoznatija ovakva filtera su bilateralni filter (eng. \emph{Bilateral filter}) i vođeni filter (eng. \emph{Guided filter}) kojim se ovaj rad detaljnije bavi u nastavku. 

\section{Vođeni filter}\label{4}%%%%%%%%%%%%%%%%%%%%%%%%

\subsection{Uvod}%%%%%%%%%%%%%%%%%%%%%%%

Veliki broj aplikacija u oblastima kompjuterske vizije i računarske grafike sadrže proces zadužen za filtriranje slika, u cilju uklanjanja i/ili izvlačenja podataka sa slike. Linearni filteri invarijantni na tranziciju, kao što su filter srednje vrednosti, \emph{Laplacinan}, Gausov i Sobelovi filteri, se široko koriste prilikom obrade slike, za restauraciju, zamućivanje i izoštravanje, detekciju ivica, izvlačenje podataka, dinamičku kompresiju itd. Ovi filteri su prostorno invarijantni i nezavisni od sadržaja slike. 

Pa ipak oni nekad ne mogu da zadovolje očekivani odgovor neke obrade. Zbog toga se ovde uvodi pojam slike vodiča \emph{(eng. guide)}, koja sadrži dodatne informacije koje mogu da se iskoriste pri filtriranju i poboljšaju rezultat. Ovaj koncept je prvi put upotrebljen u procesu difuzije \emph{(eng. anisotropic diffusion)}, u kojem gradijenti filtrirane slike služe kao vodič za ovaj proces i tako ovaj proces izbegava da uništi oštrinu ivica. Takođe WLS filter \emph{(eng. weighted least squares)} koristi ulaznu sliku kao vodič, optimizuje kvadratnu funkciju, koja je ekvivalentna procesu difuzije, koji je malopre pomenut. U mnogim primenama kao slika vodič može da se koristi različita slika u odnosu na onu koja se filtrira. Npr. kod procesa uklanjanja magle sa slike, kao slika vodič se koristi procena količine smetnji na slici, koja treba da se filtrira. Većina ovih procesa je implementirana, optimizacijom kvadratne funkcije, čiji su parametri uzeti iz slike vodiča. Rešenje je predstavljeno rešenjem velikih retko popunjenih matrica. Ovakve ne homogene matrice predstavljaju filtere koji su invarijantni u odnosu na tranziciju. Iako su ovakvi filteri efikasni, oni mogu da budu zahtevni za izvršavanje.

Drugi način da se iskoristi slika vodič je da se njene osobine eksplicitno iskažu u kernelu. \emph{Bilateral filter} je najpoznatiji primer takvog filtera. Rezultat koji piksel dobija, kod ovog filtriranja, je težinska prosečna vrednost piksela i njegovih suseda, gde težine zavise od sličnosti piksela u vodič slici. Vodič slika može i sama da bude ulazna slika u ovom filteru. \emph{Bilateral filter} može da izglača sitne fluktuacije, a da pri tome zadrži ivice. Problem ovog filtera je što ima loše artifakte koji se javljaju blizu ivica. Dolazi do promene boje blizu ivica. Takođe problem je implementirati efikasan algoritam, metodi koji ubrzavaju ovaj algoritam sa druge strane utiču na preciznost njegove obrade.

U radovima [2] i [3] se predlaže \emph{Guided filter} ili vođeni filter. Rezultat filtriranja je lokalna linearna transformacija slike vodiča. Sa jedne strane ovaj filter ima dobre osobine, pošto čuva ivice slike, a sa druge strane nema loše artifakte kao \emph{Bilateral filter}. vođeni filter može da se koristi kao filter za uglačavanje, ali njegova namena prevazilazi samo to. Uz pomoć slike vodiča rezultat ovog filtriranja će biti slika koja je manje zamućena od originalne slike i bolje struktuirana od nje. Ovaj filter se pokazao dobro u raznim primenama kao što su poboljšanje slike, uklanjanje šuma, HDR kompresija, matiranje idt. Još jedan prednost vođenog filtera je to što ima složenost $O(n)$, gde je $n$ broj piksela slike. Može da se koristi kako na jedno-kanalnim slikama tako i na više-kanalnim slikama. Prema radovima [2] i [3] ovo je najbrži filter ove namene, koji na CPU implementaciji obrađuje 1 mega piksel(1024 x 1024) za 40ms. 

\subsection{Slične ideje}\label{4.2}%%%%%%%%%%%%%%%%%%%%%%% 

\emph{Bilateral filter} je tipičan primer ovog tipa filtara i jedan od prvih koji je zamišljen. To su filteri koji služe za uglačavanje ali pri tome zadržavaju ivice slike. Rezultat filtriranja svakog piksela slike je težinska srednja vrednost piksela i njegovih suseda. Ima široku primenu. Koristi se za redukciju šuma, HDR kompresiju, dekompoziciju detalja. Generalizovan je u implementaciji \emph{joint bilateral filter}-a, gde se težine računaju u odnosu na sliku vodiča, za koju se koristi posebna slika. Ovaj filter se koristi kada se od filtriranja ne očekuje da pruži informacije o ivicama, npr. kada slika ima dosta šuma ili kada je rezultat koji treba dalje da se iskoristi. Iako je \emph{Bilateral filter} veoma popularan i intuitivan za shvatanje, on ima ograničenja. 

Prethodno spomenuti loši artifakti koji se javljaju blizu ivica, jer je Gausova funkcija nestabilna blizu ivica zbog velikih razlika u vrednostima piksela. Drugi problem se odnosi na složenost algoritma. Brute-force implementacija ima složenost $O(Nr^2)$, gde je $N$ broj piksela na slici, a $r$ je radijus kernela (dimenzija kernela je $a = 2r + 1$). Jasno je da složenost zavisi od veličine kernela što može da bude loše ako se koriste veliki kerneli. Postoje rešenja koja koriste razne metode kao što su distribuirani histogrami koji složenost smanjuju na $O(n \log{r})$. Postoji i metod za redukciju složenosti na $O(n)$ korišćenjem integralnih histograma (kao lookup tabela). Međutim konstrukcija histograma je primena filtera u prostornom domenu, pa možemo da zaključimo da ovaj proces zahteva dodatne korake. Postoji još jedan pristup čija je složenost $O(n)$, a koji se oslanja na agresivno smanjenje slike \emph{subsampling}, čime se gube detalji. Važno je naglasiti da sva spomenuta rešenja koja imaju lineranu složenost, sa druge strane utiču na degradaciju preciznosti. Pošto \emph{Bilateral filter} ima svoja ograničenja ljudi su krenuli da istražuju nove metode za filtere koji uglačavaju sliku, a pri tome čuvaju ivice. 

Nasuprot lineranim filterima postoje implicitni filteri koji se dobijaju optimizacijom funkcije greške i rešavanjem linearnog sistema jednačina, što je ekvivalentno implicitnom filtriranju slike korišćenjem inverzne matrice. Ovi filteri takođe imaju široku primenu u segmentaciji slike, kolorizaciji slike, itd. Iako su ovakvi filteri efikasni i daju kvalitetan rezultat, međutim računanje linearnih sistema može da bude veoma složeno. Direktna rešenja kao što je Gausova eliminacija, nisu praktični, zbog velikog zauzeća memorije. Iterativna rešenja sporo konvergiraju. Uočeno je da su implicitni filteri slični kao eksplicitni. 

Očuvanje ivica može da se postigne i filterima koji ne koriste srednje vrednosti susednih piksela. Najpoznatiji je medijan filter koji je objašnjen u pot-poglavlju ~\ref{3.2}.

\subsection{Definicija filtra}\label{4.3}%%%%%%%%%%%%%%%%%%%%%%%

Prvo ćemo definisati uopšteni proces linearnog invariantnog filtriranja, koji koristi sliku vodič $I$, sliku koja se filtrira $p$ i sliku koja je rezultat filtriranja $q$. $I$ i $p$ su zadati na početku procesa i oni mogu da budu identični. Rezultat filtriranja piksela $i$ će biti: 

\begin{equation}\label{eq:gf1}
q_i = \sum_{j} W_{ij}(I)p_j,
\end{equation}

gde su $i$ i $j$ indeksi piksela, kernel filtera $W_{ij}$ je funkcija slike vodiča $I$ i nezavisna je od $p$. Ovaj filter je linearan u odnosu na $p$. 

Sada ćemo definisati vođeni filter. Pretpostavka je da je vođeni filter lokalno linearan model između slike vodiča $I$ i rezultata filtriranja $q$. Za $q$ pretpostavljamo da je linearna transformacija slike $I$ u prozoru $w_k$ centriranom u pikselu $k$:

\begin{equation}\label{eq:gf2}
q_i = a_kI_i + b_k, \forall \epsilon w_k,
\end{equation}

gde su $(a_k, b_k)$ linearni koeficijenti za koje ćemo pretpostaviti da su konstantni u $w_k$. Za $w_k$ koristimo kvadratni prozor radijusa $r$. Ovaj lokalni linearni model garantuje da će $q$ imati ivicu samo ako $I$ ima tu istu ivicu, jer je $\Delta q = a\Delta I$. Da bi odredili linearne koeficijente $(a_k, b_k)$, trebaju nam ograničenja slike koja se filtrira $p$. Rezultat filtriranja određenog piksela $q_i$ se dobija tako što se od $p_i$ oduzimaju neke neželjene komponente kao što su razni detalji i šum: $q_i = p_i - n_i$. Mi tražimo rešenje koje će dati minimalnu razliku između $p$ i $q$, pod uslovom da je ta funkcija linearna. Konkretno za svaki prozor $w_k$ treba da minimizujemo sledeću funkciju greške za ~\ref{eq:gf2}:

\begin{equation}\label{eq:gf3}
E(a_k, b_k) = \sum_{i \epsilon w_k} ((a_kI_i + b_k - p_i)^2 + \varepsilon a_k^2).
\end{equation}

$\varepsilon$ je regularizacioni parametar koji služi da smanji uticaj $a_k$, ako je njena vrednost velika. Minimizacija jednačine ~\ref{eq:gf3} predstavlja problem koji se rešava linearnom regresijom. Ovde ne ulazimo u detalje ovog rešenja već dajemo rešenja direktno:

\begin{equation}\label{eq:gf4}
a_k = \dfrac{\dfrac{1}{|w|} \sum_{i \epsilon w_k} I_ip_i - \mu_k \overline{p_k}}{\sigma_k^2 + \varepsilon},
\end{equation}

\begin{equation}\label{eq:gf5}
b_k = \overline{p_k} - a_k \mu_k.
\end{equation}

$\mu_k$ i $\sigma_k^2$ su srednja vrednost i varijansa piksela slike $I$ u prozoru $w_k$, a $|w|$ je broj piksela u $w_k$, $\overline{p_k} = \dfrac{1}{|w|} \sum_{i \epsilon w_k} p_i$ je srednja vrednost piksela slike u $w_k$. Kada smo sračunali linearne koeficijente $(a_k, b_k)$, možemo da nastavimo sa postupkom za izračunavanje rezultata $q_i$. Pošto je će piksel $i$ biti deo različitih prozora $w_k$, prilikom filtriranja, ne možemo da koristimo jedinstvene koeficijente $(a_k, b_k)$ za jedan piksel. Zbog toga ćemo u obzir uzeti sve moguće vrednosti koeficijenata za $q_i$. Kada izračunamo vrednosti $(a_k, b_k)$ za svaki piksel ulazne slike, rezultat filtriranja će biti:

\begin{equation}\label{eq:gf6}
q_i = \dfrac{1}{|w|} \sum_{k|i \epsilon w_k} (a_kI_i + b_k).
\end{equation} 

Pošto su prozori $w_k$ kvadratnog oblika i simetrični su, možemo da primetimo da je $\sum_{k|i \epsilon w_k} a_k = \sum_{k \epsilon w_i} a_k$, pa formula ~\ref{eq:gf6} može da se napiše ovako:

\begin{equation}\label{eq:gf7}
q_i = \overline{a_i}I_i + \overline{b_i},
\end{equation}

gde su $\overline{a_i} = \dfrac{1}{|w|}\sum_{k \epsilon w_i} a_k$ i $\overline{b_i} = \dfrac{1}{|w|}\sum_{k \epsilon w_i} b_k$ srednje vrednosti iz svih prozora u kojima je $q_i$ sadržan. Sa promenom koja je uvedena u ovoj jednačini gubimo sličnost između $\Delta q$ i $\Delta I$ na mestima gde su ivice, jer linearni koeficijenti $(\overline{a_i}, \overline{b_i})$ variraju u zavisnosti od prostora. Međutim pošto su oni rezultat filtera srednje vrednosti, možemo da očekujemo da će njihovi graijenti biti mnogo manji nego gradijenti $I$ blizu jakih ivica. U ovoj situaciji možemo da kažemo da je $\Delta q \approx \overline{a} \Delta I$, što znači da će nagle promene intenziteta na $I$ biti sačuvane i na $q$. 


\subsection{Očuvanje ivica}%%%%%%%%%%%%%%%%%%%%%%%

S obzirom da je jedan od glavnih prednosti ovog filtera očuvanje ivica, ovde ćemo proveriti tu osobinu. Na sledećoj slici su prikazani primeri filtriranja vođenim filterom, korišćenjem različitih parametara. Ovo je primer gde se koristi identična slika i kao ulaz i kao slika vodič. Na ovom primeru možemo da vidimo da se vođeni filter ponaša kao filter koji čuva ivice, iako je njegova namena da uglača sliku. 
Na primerima koji su prikazani na slici 13 možemo da uočimo ove osobine i možemo da primetimo razliku između vođenog filtera i filtara prikazanih u pot-poglavlju ~\ref{3.2} koji služe za uglačavanje slike.

\begin{figure}[ht!]
\centering
\includegraphics[width=60mm]{img/img.png}
\includegraphics[width=60mm]{img/imgGF2_01.png}
\includegraphics[width=60mm]{img/imgGF4_01.png}
\includegraphics[width=60mm]{img/imgGF2_16.png}
\includegraphics[width=60mm]{img/imgGF8_04.png}
\includegraphics[width=60mm]{img/imgGF4_16.png}
\caption{Primena vođenog filtera: a) slika fotografa; Slike filtrirane vođenim filterom sa parametrima b) $r = 2$ $\varepsilon = 0.01$; c) $r = 4$ $\varepsilon = 0.01$; d) $r = 2$ $\varepsilon = 0.16$; e) $r = 8$ $\varepsilon = 0.04$; f) $r = 4$ $\varepsilon = 0.16$}
\label{gfilter}
\end{figure}

Sa slike ~\ref{gfilter} možemo da zaključimo da regularizacioni parametar $\varepsilon$ utiče na to koliko će slika biti zamućena odnosno šta je na slici ivica koja trba da se sačuva, a radijus prozora $r$ utiče na to koliko će suseda biti korišćeno pri filtriranju, što će kao kod filtara za uglačavanje uticati na količinu uklonjenog šuma. 

Osobina očuvanja ivice može intuitivno da se objasni ovako, ako uzmemo da je $I \equiv p$. U ovom slučaju će $a_k$ i $b_k$, koji su definisani u pot-poglavlju ~\ref{4.3}, da izgledaju ovako $a_k = \sigma_k^2 / (\sigma_k^2 + \varepsilon)$, $b_k = (1 - a_k)\mu_k$. Jasno je da kada je $\varepsilon = 0$ onda je $a_k = 1$ i $b_k = 0$, pa će $q = I$. Kada je $\varepsilon > 0$ onda treba da razmotrimo dva slučaja. Prvi je slučaj ''visoke varijanse'', kada $I$ ima dosta promena vrednosti unutar prozora $w_k$, tada je $\sigma_k^2 >> \varepsilon$, pa je $a_k \approx 1$ i $b_k \approx 0$. Drugi je slučaj ''ravnog područija'', kada $I$ ima skoro identične vrednosti u prozoru $w_k$, tada je $a_k \approx 0$ i $b_k \approx \mu_k$. 

Kada računamo srednje vrednosti za $a_k$ i $b_k$ da bi dobili $\overline{a_i}$ i $\overline{b_i}$, koje koristimo da bi dobili rezultat (kao što je definisano u pot-poglavlju ~\ref{4.3}), tada će, ako se piksel nalazi u prozoru ''visoke varijanse'', njegova vrednost biti nepromenjena ($a \approx 1, b \approx 0, q \approx p$). U slučaju da se piksel nalazi u oblasti ''ravnog područija'', tada će rezultujuća vredost biti srednja vrednost piksela suseda ($a \approx 0, b \approx \mu, q \approx \overline{\mu}$). Ova dva slučaja su određena parametrom $\varepsilon$. Delovi slike, gde je varijansa $\sigma^2$ mnogo manja od $\varepsilon$, su uglačana odnosno smooth-ovana, a mesta gde je $\sigma^2$ mnogo veća od $\varepsilon$ su očuvana. U suštini parametar regularizacije $\varepsilon$ određuje šta je ivica, odnosno šta treba da ostane nepromenjeno. 

\subsection{Filtriranje slike u boji}%%%%%%%%%%%%%%%%%%%%%%% 

Princip primene ovog filtera i definicija može lako da se proširi za primenu nad više-kanalnim slikama. Za nas su od interesa slike koje su prikazane sa tri kanala odnosno slike u boji. U zavisnosti od primene ovo može da se reši na više načina. Najprostije rešenje je da se definicija iz 3.3 primeni na sva tri kanala zasebno, a da se pri tome za sliku vodiča koristi gray-scale (jedno-kanalna) verzija originalne slike. Ovo u većini slučaja može da da zadovoljavajuće rezultate. 

Međutim u nekim slučajevima slika vodič u boji (tro-kanalna slika) može bolje da sačuva ivice. Naime prilikom konvertovanja u gray-scale sliku dve različite boje predstavljene sa tri vrednosti (najčešće RGB model) mogu da imaju sličnu vrednost u gray-scale slici. U ovom slučaju će doći do pojave takozvanog ''halo'' efekta, tj. ivice će početi da se smooth-uju. Zbog toga je u ovakvim slučajevima koristiti sliku vodič u boji a proširenje modela ~\ref{eq:gf2} definisanog u pot-poglavlju ~\ref{4.3} se postiže ovako:

\begin{equation}\label{gf10}
q_i = a_k^T I_k + b_k, \forall i \epsilon w_k,
\end{equation} 

gde je $I_i$ 3 x 1 vektor boja, $a_k$ 3x 1 vektor koeficijenata, a $q_i$ i $b_k$ su skalari. Formule ~\ref{eq:gf4}, ~\ref{eq:gf5}, ~\ref{eq:gf7} definisane u 3.3 će postati:

\begin{equation}\label{eq:gf11}
a_k = (\Sigma_k + \varepsilon U)^{-1} (\dfrac{1}{|w|} \sum_{i \epsilon w_k} I_i p_i - \mu_k \overline{p_i}),
\end{equation}

\begin{equation}\label{eq:gf12}
b_k = \overline{p_k} - a_k^T \mu_k,
\end{equation}

\begin{equation}\label{eq:gf13}
q_i = \overline{a_i^T} I_i + \overline{b_i},
\end{equation}

gde je $\Sigma_k$ 3 x 3 matrica kovarijanse slike $I$ u prozoru $w_k$ i $U$ je 3 x 3 matrica idnetiteta.

\subsection{Izračunavanje i efikasnost}%%%%%%%%%%%%%%%%%%%%%%% 

Glavna prednost vođenog filtera u odnosu na slične filtere kao što je bilateralni filter, koji je spomenut u pot-poglavlju ~\ref{4.2}, se ogleda u tome što ima složenost $O(n)$. Njegova složenost je nezavisna od radijusa prozora (kernel) koji se koristi za filtriranje i od intenziteta filtriranja. Kod bilateralnog filtera složenost, kao što je pokazano u pot-poglavlju ~\ref{4.2}, zavisi od veličine prozora. Umesto da direktne primene konvolucije, kao kod većine konvolucionih filtara, ovde se vrednost rezultujućeg piksela računa kao linearna funkcija definisana u 3.3. Prvo se računaju koeficijenti $a_k$ i $b_k$ u formulama ~\ref{eq:gf4} i ~\ref{eq:gf5}. U tim formulama možemo da primetimo da postoji nekoliko sumiranja: $\sum_{i \epsilon w_k} I_ip_i, \mu_k$ je srednja vrednost, $\sigma_k^2$ je varijansa, $\overline{p_k}$ srednja vrednost piksela. Srednja vrednost i varijansa takođe zahtevaju izračunavanje sume odgovarajućih piksela. Ove sume možemo da izračunamo u $O(1)$ korišćenjem \emph{look-up} tabela takozvanih integralnih slika. Kada se izračunaju koeficijenti $a_k$ i $b_k$, onda se rezultat dobija iz formule ~\ref{eq:gf7}, 
gde su $\overline{a_i} = (1 / |w|) \sum_{k \epsilon w_i} a_k$ i $\overline{b_i} = (1 / |w|) \sum_{k \epsilon w_i} b_k$ srednje vrednosti iz svih prozora u kojima je $q_i$ sadržan. Ove dve sume takođe možemo da izračunamo u $O(1)$ korišćenjem integralnih slika. Pošto je složenost računanja vrednosti za jedna piksel $O(1)$ možemo da zaključimo da će složenost algoritma biti $O(n)$, gde je $n$ broj piksela.

Integralna slika je specijalni slučaj ''lookup'' tabele koja se još naziva i tablica suma. Tablica suma je struktura podataka koja služi za efikasno računanje suma pravougaonastog podskupa neke mreže. U slučaju obrade slike ona predstavlja matricu koja je istih dimenzija kao slika. U elementu matrice koji se nalazi u vrsti $x$ i koloni $y$, koji se odnosi na piksel sa koordinatama $(x, y)$, se čuva suma svih elemenata koji imaju manju $x$ i $y$ koordinatu od tog elementa. 

\begin{equation}\label{eq:gf17}
I(x, y) = \sum_{x' <= x, y' <= y} i(x', y'),
\end{equation}

gde je $I(x, y)$ vrednost u tablici, a $i(x, y)$ je vrednost piksela. Izračunavanje ove tablice je efikasno i može da se izračuna u jednom prolazu kroz sliku (O(n)):

\begin{equation}\label{eq:gf18}
I(x, y) = i(x, y) + I(x, y - 1) + I(x - 1, y) - I(x - 1, y - 1).
\end{equation}

\begin{equation}\label{eq:gf18}
A
=
\begin{bmatrix}
1 & 2 & 3 \\
4 & 5 & 6 \\
7 & 8 & 9 \\
\end{bmatrix}
sat(A)
=
\begin{bmatrix}
1 & 3 & 6 \\
5 & 12 & 21 \\
12 & 27 & 45 \\
\end{bmatrix}
.\end{equation}

Kada se tablica $I$ izračuna, suma svake pod-slike može da se izračuna čitanjem vrednosti u tačkama pravougaonika koji formira ta pod-slika. Ako su ovo tačke koje formiraju pravougaonik $A = (x_0, y_0), B = (x_1, y_0), C = (x_0, y_1)$ i $D = (x_1, y_1)$ suma te pod-slike se računa kao:

\begin{equation}\label{eq:gf19}
sum = \sum_{x_0 < x <= x_1, y_0 < y <= y_1} i(x, y) = I(D) + I(A) - I(B) - I(C).
\end{equation} 

\begin{figure}[ht!]
\centering
\includegraphics[width=60mm]{img/sat.png}
\caption{Izračunavanje sume pod-slike}
\label{overflow}
\end{figure}

U radu [3] su vršeni eksperimenti na računaru sa procesorom intel core i7 brzine 3GHz i sa 8GB RAM memorije. Implementacija u C++, bez paralelizacije. Vreme izvršenja ovog algoritma za jedno-kanalnu sliku je 40 ms/Mp (40 milisekundi po megapikselu), što je mnogo brže od bilo koje druge implementacije sličnih filtera. 

\subsection{Poboljšanje}%%%%%%%%%%%%%%%%%%%%%%% 

U radu [3], istih autora kao i [2] i [3] gde je definisan vođeni filter, je predloženo ubrzanje algoritma filtera koji je definisan u pot-poglavlju ~\ref{4.3}. Autori su primetili strategiju koja bi im koristila da ubrzaju ovaj algoritam, koja je bila vezana za ponašanje koeficijenata prilikom povećanja i smanjenja slike. Pod povećanjem \emph{eng. upsample} i smanjenjem \emph{eng. downsample} se podrazumevaju postupci kod kojih se slika transformiše u sliku veće rezoluciije (povećanje broja piksela) ili se transformiše u sliku manje rezolucije (smanjenje broja piksela). Ideja je da se računanje koeficijenata koji se koriste u izračunavanju vrednosti rezultujućeg piksela ($a_k, b_k$ zatim $\overline{a_i}, \overline{b_i}$) obavi na downsample-ovanoj slici zatim da se ti koeficijenti iskoriste za izračunavanje vrednosti. Takva promena bi smanjila složenost algoritma na $O(N / s^2)$, gde je $s$ faktor smanjenja, a $N$ broj piksela na slici. 

Ulazna slika $p$ i slika vodič $I$ se smanjuju. Zatim se koeficijenti $a_k, b_k$, pa nakon toga i $\overline{a_i}, \overline{b_i}$ računaju na umanjenoj slici. Nakon toga se slike koje sadrže koeficijente $\overline{a_i}, \overline{b_i}$ uvećavaju na veličinu ulazne slike i onda se ti koeficijenti primenjuju na odgovarajuće piksele ulazne slike $p$. Sva izračunavanja i računanje suma, za koje se formiraju tabele suma, računaju se na umanjenoj slici.

Kompleksnost računanja se smanjuje sa $O(n)$ na $O(n / s^2)$. U radu [5] je eksperimentima utvrđeno da se odabirom odgovarajućeg faktora $s$, dobijaju nepromenjeni rezultati. U ovim eksperimentima je utvrđeno da ako je $s = 4$ ubrzanje može da bude i 10 puta. Odabirom odgovarajućeg faktora može da se dobije zadovoljavajući rezultat. 

\subsection{Ograničenja}%%%%%%%%%%%%%%%%%%%%%%%

Ograničenja vođenog filtera se ogledaju u tome što može da dođe do pojave ''halo'' efekta. ''Halo'' efekat se odnosi na artifakte neželjenog uglačavanja ivica. Ovi efekti su neizbežni za lokalne linearne filtere. Ako slika sadrži jake teksture koje želimo da uglačamo tj. da uklopimo u okolinu, moramo da primenimo vođeni filter sa velikim parametrima. Ovo će rezultovati uglačavanjem ivica koje nisu preterano izražene (promena vrednosti piksela nije velika). U ovom slučaju će se javiti isti artifakti kao kod primene filtera srednje vrednosti. O ovom ograničenju će biti reči poglavlju ~\ref{5}.

Ipak i pored ovog ograničenja vođenog filter je našao veliku primenu u svetu kompjuterskog vida i grafike, jer daje dobre rezultate kad je u pitanju očuvanje ivica i uklanjanje šuma. A pri tome je efikasan za izvršavanje. Pošto lokalni model, koji se koristi za vođeni filter, tip ne nadgledanog učenja, ostavlja se prostor za primenu poboljšanih modela kako bi se stvorili novi filteri.

\section{Primena}\label{5}%%%%%%%%%%%%%%%%%%%%%%%%%%

\subsection{Osnovna primena}%%%%%%%%%%%%%%%%%%%%%%%

Vođeni filter može da se koristi isto kao \emph{smoothnig} filteri opisani u poglavlju ~\ref{3}. Možemo da ga koristimo u svrhu uglačavanja slike ili uklanjanja šuma. Naravno prednost vođenog filtera je što obraća pažnju na detalje i ivice na slici. Na slici je ~\ref{flowerGF} prikazan rezultat filtriranja slike, sa tri različita filtera. Korišćeni su vođeni filter i filteri koji su prikazani u poglavlju ~\ref{4}, filter srednje vrednosti i medijan filter. 

\begin{figure}[ht!]
\centering
\includegraphics[width=35mm]{img/flower.png}
\includegraphics[width=35mm]{img/flowerGF30_01.png}
\includegraphics[width=35mm]{img/flowerAvg30.png}
\includegraphics[width=35mm]{img/flowerMed30.png}
\caption{Primer uglačavanja slike korišćenjem raznih filtara: a) originalna slika b) vođeni filter $r = 30$ $\varepsilon$ = 0.01 c) filter srednje vrednosti veličina prozora 30 d) medijan filter veličina prozora 30 }
\label{flowerGF}
\end{figure}

Na slici ~\ref{flowerGF} možemo da vidimo prednost vođenog filtera u odnosu na filter srednjih vrednosti koji ne vodi računa o ivicama, koje su u ovom primeru uglačane. Možemo da vidimo i da medijan filter daje malo bolje rezultate od vođenog filtera. Problem je što medijan filter ima veliku složenost, jer za svaki piksel rezultujuće slike sortiraju svi pikseli koji pripadaju prozoru filtriranja.

U tabeli ~\ref{tabela} je dato poređenje vremena izvršenja ova tri filtera za slike različitih dimenzija, sa različitim dimenzijama prozora koji se koriste za filtriranje. Vremena u ovoj tabeli su vremena izmerena na izvršenju matlab skripti. Vreme izvršenja je sigurno manje, ako koristimo c++ implementaciju, vremena koja su prikazana u tabeli ~\ref{tabela} služe da pokažu odnos u izvršenju ovih filtara. Na osnovu vremena iz tabele ~\ref{tabela} možemo da zaključimo da vođeni filter ima manje vreme izvršenja od medijan filtera i takođe možemo da se vreme izvršenja vođenog filtera ne menja, kada se menja radijus prozora za filtriranje, što nije slučaj za druga dva filtera. 

\begin{table}[h!]
\centering
\begin{tabular}{| c | c c c|} 
\hline
Dimenzije slike i radijusa prozora [r] & Vođeni filter & Filter srednjih vrednosti & medijan filter\\ [0.5ex] 
\hline
276x276p r = 10 & 13.4 & 1.8 & 173.7 \\ 
276x276p r = 20 & 13.5 & 3.1 & 573.2 \\
276x276p r = 30 & 13.3 & 3.9 & 1211.5 \\
512x512p r = 10 & 95.6 & 7.9 & 598.4 \\
512x512p r = 20 & 97.7 & 9.9 & 1864 \\
512x512p r = 30 & 95.7 & 12.2 & 3790.7 \\
1200x800p r = 10 & 361.9 & 27 & 1856 \\
1200x800p r = 20 & 381.9 & 35.5 & 5990.3 \\
1200x800p r = 30 & 389 & 44.3 & 12664 \\ [1ex] 
\hline
\end{tabular}
\caption{Poređenje vremena izvršenja vođenog filtera, filtera srednjih vrednosti i medijan filtera. Vremena su izražena u mili-sekundama [ms]. U tabeli su prikazana prosečna vremena izvršenja ovih filtera na slikama različitih rezolucija, a za svaku rezoluciju su ispitivana vremena izvršenja filtera ako se koriste različite vrednosti radijusa prozora koji se koristi za filtriranje. }
\label{tabela}
\end{table}

U primeru na slici ~\ref{flowerGF} se koristi ista slika i kao ulazna i kao slika vodič. Ovde možemo da uočimo dobre strane vođenog filtera. Vidimo da je šum uklonjen, a pri tome su sačuvane ivice odnosno prelazi. Ovde je iskorišćena proširena verzija koja uzima u obzir sva tri kanala, prilikom obrade ivica. Ako se koristi pristup u kome se filtrira svaki kanal zasebno korišćenjem odgovarajućeg kanala slike vodiča može doći do pojave loših ivica. 

\begin{figure}[ht!]
\centering
\includegraphics[width=40mm]{img/zoom1.png}
\includegraphics[width=40mm]{img/zoom2.png}
\caption{Razlika između slike koja koristi jedno-kanalnu sliku kao vodič i slike koja koristi tro-kanalnu sliku kao vodič a) tro-kanalna slika vodič b) jedno-kanalna slika vodič }
\label{overflow}
\end{figure}

Jedna od prostijih, ali veoma korisnih primena je poboljšanje slike dodavanjem detalja na sliku. Ako je $I$ slika koju filtriramo, a $J$ rezultat koji želimo da postignemo, ovaj filter možemo da predstavimo kao:

\begin{equation}\label{eq:enh}
(x, y) = I(x, y) + (I(x, y) - guidedFilter(I(x, y)) * A),
\end{equation}

gde je A parametar kojim kontrolišemo koliko hoćemo da nam detalj bude izražen. Rezultate primene možemo da vidimo na slici ~\ref{enha}.

\begin{figure}[ht!]
\centering
\includegraphics[width=40mm]{img/flower.png}
\includegraphics[width=40mm]{img/flowerEGF.png}
\includegraphics[width=40mm]{img/flowerER.png}
\caption{Poboljšanje slike uz pomoć vođenog filtera: a) originalna slika b) primenjen vođen filter $r = 16$ $\varepsilon = 0.01$ c) rezultat za parametar iz jednačine ~\ref{eq:enh} A = 5 }
\label{enha}
\end{figure}

Na slici ~\ref{enha1} možemo da vidimo razlike između korišćenja različitih filtara za poboljšanje slike definisano formulom ~\ref{eq:enh}. Možemo da zaključimo da je primer gde je korišćen vođeni filter pokazao bolje rezultate u naglašavanju detalja.

\begin{figure}[ht!]
\centering
\includegraphics[width=40mm]{img/flowerER.png}
\includegraphics[width=40mm]{img/flowerERAvg.png}
\caption{Razlika u poboljšanju slike vođenog filtera i filtera srednjih vrednosti: a) primenjen vođen filter $r = 16$ $\varepsilon = 0.01$ b) primenjen filter srednjih vrednosti $r = 16$ }
\label{enha1}
\end{figure}

\newpage
\subsection{Uklanjanje izmaglice sa slike (''Dehazing'')}%%%%%%%%%%%%%%%%%%%%%%%

\emph{Dehzing} je metoda koja služi za uklanjanje posebnog tipa šuma (haze) sa slike. \emph{Haze} predstavlja tip smetnji koje se javljaju na slikama, a posledica su atmosferskih prilika. U suštini to je izmaglica koja može da bude rezultat pojave prave magle ili vremenskih prilika. Kada se slika pravi na kišnom vremenu objekti u daljini mogu delovati zamagljeno. Takođe haze može da budu posledica isparavanja, kako u prirodi tako i industrijskog, ili dim koji je posledica požara. Cilj ove metode je da ukloni ove smetnje i istakne neke bitne detalje.

U ovom radu se predlaže implementacija algoritma za dehazing korišćenjem vođenog filtera. Prednost vođenog filtera se ogleda u tome što on ima manju složenost u odnosu na druge filtere, pa možemo da zaključimo da će ta implementacija biti najbrža. Za potrebe ovog rada napravljena je implementacija u programskom jeziku Matlab. Implementacija je napravljena korišćenjem ideja iz rada [6]. 

U radu [6] je dat predlog za rešavanje ovog problema. Pretpostavka je da svaki deo slike ima tamne piksele, odnosno u svakom delu slike može da se pronađe piksel čiji je jedan kanal približno jednak nuli, ako se zanemari \emph{haze}. U takvom pikselu će vrednost predstavljati vrednost količine \emph{haze}-a. Ako se odabere pod-slika odgovarajuće veličine količina procenjenog šuma na tom pikselu može da se generalizuje na celu tu pod sliku. Na ovaj način može da se proceni količina šuma celoj slici. Na kraju se od svakog piksela biti oduzeta vrednost \emph{haze}-a procenjenog za taj piksel i na taj način se dobija filtrirana slika.

Model slike koji je dat u [6] izlgeda ovako:

\begin{equation}\label{eq:haze1}
I(x) = J(x)t(x) + A(1 - t(x)),
\end{equation}

gde je $I$ posmatrana slika, $J$ vektor površinskog odsijavanja na tački preseka scene i zraka iz realnog sveta, koja odgovara pikselu $x = (z, y)$, a $t(x)$ je transmisija duž zraka. $A$ je atmosferska osvetljenost odnosno osvetljenost prostora scene. Ova definicija važi za sva tri RGB kanala. Ovo se drugačije naziva i model degradacije i koristi se za predstavljanje slike koja sadrži smetnje kao što je magla. 

Ono što mi tražimo je $J$ tj. model tamnog kanala koji predstavlja mračne kanale na slici, da bi odredili količinu magle (\emph{haze}) na njoj. Model za formiranje tamnog kanala je sledeći:

\begin{equation}\label{eq:haze2}
J^{dark}(x) = min_{c \epsilon \{r, g, b \}}( min_{y \epsilon \Omega (x)} (J^c (y)) ),
\end{equation}

gde je se u okviru piksela i njegovih suseda definisanih prozorom $\Omega(x)$ traži piksel koji u nekom od RGB kanala ima minimalnu vrednost. Za svaki piksel ulazne slike se računa minimalna vrednost za sve piksele u prozoru veličine $2*r + 1$, po sva tri kanala. Pošto za svaki piksel moramo da obradimo njegove susede u prozoru veličine $2*r + 1$ možemo da zaključimo da će složenost ovog izračunavanja biti $O(Nr^2)$. Na slici ~\ref{TamniKanal} možemo da vidimo rezultate izračunavanja tamnog kanala i uticaj promene veličine $r$ na rezultat. Na slici ~\ref{TamniKanal} možemo da uočimo da veličina prozora koji se koristi u proceni tamnog kanala utiče na robusnost procene. 

\begin{figure}[ht!]
\centering
\includegraphics[width=45mm]{img/hazeDC0.png}
\includegraphics[width=45mm]{img/hazeDC2.png}
\includegraphics[width=45mm]{img/hazeDC5.png}
\includegraphics[width=45mm]{img/haze.png}
\caption{Izračunavanje tamnog kanala na primeru zamagljene slike zgrade korišćenjem različitih veličina prozora koji se koriste u proceni vrednosti tamnog kanala; tamni kanal izračunat sa veličinom prozora a) 0 b) 2 c) 5 d) zamagljena slika zgrade}
\label{TamniKanal}
\end{figure} 

Ako formulu ~\ref{eq:haze2} primenimo u formulu ~\ref{eq:haze1} modela slike i ako pretpostavimo da je $J(x)$ za sliku bez smetnji jednaka nuli, za transmisionu sliku dobijamo da je jednaka:

\begin{equation}\label{eq:haze3}
t(x) = 1 - \omega min_c ( min_{y \epsilon \Omega (x)} (\dfrac{I^c(y)}{A^c})),
\end{equation}

gde je $\omega$ regularizacioni parametar koji određuje u kolikoj meri od 0-100 \% verujemo proceni, vrednost je minimum u prozoru $\Omega(x)$ i po sva tri kanala. Kada izračunamo sliku transmisije rezultat se na osnovu formule ~\ref{eq:haze1} dobija kao:

\begin{equation}\label{eq:haze3}
H(x) = \dfrac{I(x) - A}{t(x)} + 1.
\end{equation}

Pošto je procena transmisione komponente $t(x)$ robusna, tj procena vrednosti se vrši u prozoru, dobijeni rezultat može da utiče na ivice. Da bi se dobila finija procena koja obraća pažnju na ivice na slici, transmisiona slika se filtrira. Za ovo filtriranje može da se koristi vođeni filter. Na slici ~\ref{FiltriraniTamniKanal} možemo na primeru slike zgrade da vidimo kakav će uticaj imati filtriranje slike tamnog kanala. Na ovoj slici se vidi da filtriranje tamnog kanala, koji je procenjen korišćenjem prozora radijusa 5, utiče na robusnost te procene. Filtriranje treba da postigne da se oblasti piksla koji se nalaze na malo razdaljini izjednače, jer je pretpostavka da oni imaju jednaku količinu magle. Pri tome vođeni filter će voditi računa o oblastima koje su odvojene ivicama. 

\begin{figure}[ht!]
\centering
\includegraphics[width=60mm]{img/haze.png}
\includegraphics[width=60mm]{img/hazeDC5F.png}
\caption{Uticaj filtriranja slike tamnog kanala na slici zgrade a) tamni kanal sa prozorom dimenzije 5 b) filtrirani tamni kanal sa prozorom dimenzije 5}
\label{FiltriraniTamniKanal}
\end{figure} 

U implementaciji čiji su rezultati biti prikazani u ovom radu se koristi sledeći postupak za dobijanje transmisione slike. Pretpostavka je da je svetlost prostora $A = (1, 1, 1)$ i da je $t(x) = const.$ u okviru prozora koji se koristi za filtriranje. Postupak je opisan redom formulama ~\ref{eq:haze4}, ~\ref{eq:haze5}, ~\ref{eq:haze6} i ~\ref{eq:haze7}:

\begin{equation}\label{eq:haze4}
A(x) = guidedFilter(J^{dark}(x)).
\end{equation}

\begin{equation}\label{eq:haze5}
B(x) = A(x) - guidedFilter(|J^{dark}(x) - A(x)|).
\end{equation}

\begin{equation}\label{eq:haze6}
V(x) = 
\begin{cases}
\text{0,} &\quad\text{B(x) <= 0}\\
\text{1,} &\quad\text{B(x) > 1} \\
\text{B(x),} &\quad\text{u ostalim slučajevima}\\
\end{cases}
. \end{equation}

\begin{equation}\label{eq:haze7}
t(x) = 1 - w * V(x).
\end{equation}

Prvo se filtrira robusna procena $J^{dark}$, uz pomoć vođenog filtera, da bi dobili prvi rezultat $A(x)$. Filtriranje u drugom koraku se vrši sa ciljem da se poboljšaju informacije o ivicama na slici. Ovo se postiže filtriranjem apsolutne razlike $|J^{dark}(x) - A(x)|$ koja će nositi informaciju o ivicama. Primenom ovih transformacija može doći do pojavljivanja vrednosti koje prevazilaze granice domena pa se vrši normalizacija u trećem koraku i dobijamo $V(x)$, gde su sve vrednosti iz domena. Pretpostavka je da su vrednosti iz domena $(0, 1)$ (ekvivalentno se primenjuje za bilo koji drugi domen, ali je poželjno da se koriste vrednosti razlomljenih brojeva predstavljenih u pokretnom zarezu \emph{float}). Transmisiona slika se dobija kao inverzna slika od $V(x)$, koja je pomnožena koeficijentom $\omega$, koje predstavlja koeficijent verovanja u procenu.

\begin{figure}[ht!]
\centering
\includegraphics[width=60mm]{img/haze.png}
\includegraphics[width=60mm]{img/hazeRes.png}
\caption{Rezultat filtriranja slike zgrade u cilju uklanjanja magle 1) orginalna slika 2) rezultat filtriranja}
\label{dehaze}
\end{figure} 

Na slici ~\ref{dehaze} možemo da primetimo da rezultat nije perfektan. Naime možemo da primetimo da se magla zadržala oko ivica, tj. prelaza na slici (između mesta sa različitim bojama, konkretno cigle i lišće na slici 17). Ovo je posledica ''halo'' efekta koji je jedno od loših strana vođenog filtera. Pošto transmisiona slika sadrži prelaze koji nisu izraziti (pikseli sa jedne i druge strane ivice se ne razlikuju drastično), dolazi do ispoljavanja ovog efekta.

Rezultat implementacije pokazane u ovom radu može da se menja podešavanjem parametara. Dobrim podešavanjem parametara rezultat može promeni u skladu sa različitim očekivanjima. Parametri koji mogu da se menjaju u ovoj implementaciji su:

\newpage
\begin{itemize}
\item $r_{dc}$ - radijus prozora za izračunavanje tamnog kanala 
\item $r_{gf}$ - radijus prozora za izračunavanje vođenog filtra
\item $\varepsilon_{gf}$ - regularizaciona konstanta za vođeni filter 
\item $w$ - koeficijent verovanja procene magle
\end{itemize}

Prvo ćemo ispitati uticaj parametra $r_{dc}$ koji predstavlja radijus prozora za izračunavanje tamnog kanala. Na slici ~\ref{UticajRdc} možemo da vidimo rezultate za fiksirane parametre vođenog filtera. Ovde možemo da utvrdimo da se povećanjem parametra $r_{dc}$ postiže bolji rezultat. Boje na slici sa većim parametrom $r_{dc}$ su približnije realnim bojama, međutim dolazi do povećanja ''halo'' efekta što je posledica korišćenja vođenog filtera. 

\begin{figure}[ht!]
\centering
\includegraphics[width=45mm]{img/hazeResDC0.png}
\includegraphics[width=45mm]{img/hazeResDC3.png}
\includegraphics[width=45mm]{img/hazeResDC5.png}
\caption{Rezultati filtriranja slike zgrade u cilju uklanjanja magle u kojima je menjan parametar radijus prozora za izračunavanje tamnog kanala $r$. Parametri vođenog filtera su $r_{gf} = 2$ i $\varepsilon_{gf} = 0.01$ a) $r = 0$ b) $r = 3$ c) $r = 5$}
\label{UticajRdc}
\end{figure} 

Sada ćemo ispitati uticaj parametara vođenog filtera $r_{gf}$ koji predstavlja radijus prozora i $\varepsilon_{gf}$ koji predstavlja regularizacionu konstantu za vođeni filter. Na slici ~\ref{UticajGF} možemo da vidimo rezultate. Ovde kao i u ispitivanju parametra radijusa prozora tamnog kanala, možemo da utvrdimo da se povećanjem parametra vođenog filtera postiže bolji rezultat. Boje na slici sa većim parametrima vođenog filtera su približnije realnim bojama, međutim dolazi do povećanja ''halo'' efekta što je posledica korišćenja tog filtera. 

\begin{figure}[ht!]
\centering
\includegraphics[width=45mm]{img/hazeResGF2_01.png}
\includegraphics[width=45mm]{img/hazeResGF5_04.png}
\includegraphics[width=45mm]{img/hazeResGF10_1.png}
\caption{Rezultati filtriranja slike zgrade u cilju uklanjanja magle u kojima su menjani parametri vođenog filtera $r$ i $\varepsilon$. Parametar radijusa prozora za izračunavanje tamnog kanala je $r_{dc} = 0$ a)$r = 2$ $\varepsilon = 0.01$ b)$r = 5$ $\varepsilon = 0.04$ c)$r = 10$ $\varepsilon = 0.1$}
\label{UticajRdc}
\end{figure} 

Možemo da zaključimo da se povećanjem parametara $r_{dc}$ (radijus prozora za izračunavanje tamnog kanala), $r_{gf}$ i $\varepsilon$ (parametri vođenog filtera), dolazi do povećanja kvaliteta rezultata (u smislu vizuelne recepcije i boja), ali je loša strana pojavljivanje ''halo'' efekta na mestima gde se boje menjaju (ivice slike). Možemo da zaključimo da je pametno birati manje parametre, da bi izbegli loše strane vođenog filtera.

Kod slika koje predstavljaju scenu na kojoj se javlja deo neba koji je beo od posledice izmaglice, može doći do pogrešne procene kanala. Greška će se javiti zbog velike količine svetle boje, koja će biti procenjena kao magla. Ova greška će proizvesti rezultat kao na slici ~\ref{Šuma1}.

\begin{figure}[ht!]
\centering
\includegraphics[width=60mm]{img/forest.png}
\includegraphics[width=60mm]{img/forest100.png}
\caption{Rezultati filtriranja slike šume u cilju uklanjanja magle. a) originalna slika b) filtrirana slika sa parametrom verovanja procene $w = 1$}
\label{Šuma1}
\end{figure} 

Na ovu grešku u proceni možemo da utičemo promenom parametra $w$ (koeficijent verovanja procene). Ako smanjimo vrednost parametra, kao na slici ~\ref{Šuma2}.

\begin{figure}[ht!]
\centering
\includegraphics[width=60mm]{img/forest.png}
\includegraphics[width=60mm]{img/forest95.png}
\caption{Rezultati filtriranja slike šume u cilju uklanjanja magle. a) originalna slika b) filtrirana slika sa parametrom verovanja procene $w = 0.95$}
\label{Šuma2}
\end{figure} 

Cilj korišćenja vođenog filtera u našoj implementaciji je ubrzanje algoritma. Metodi koji se koriste u [5] se oslanjaju na konvolucione filtere i izračunavanja koja su kvadratne složenosti jer zavise od prozora koji se koriste za filtriranje. Složenost vođenog filtera zavisi samo od broja piksela što ga čini bržim. Pri tome vođenog filter daje približno dobra rešenja. Na slikama ~\ref{aerial} i ~\ref{grad}. 

\begin{figure}[ht!]
\centering
\includegraphics[width=60mm]{img/aerial.png}
\includegraphics[width=60mm]{img/aerialFattal.png}
\includegraphics[width=60mm]{img/aerialDe.png}
\caption{Poređenje rezultata dehazing-a iz rada [5] (''Fattal'') i naše implementacije koja koristi vođeni filter. a) originalna slika b) ''Fattal'' c) vođeni filter}
\label{aerial}
\end{figure} 

\begin{figure}[ht!]
\centering
\includegraphics[width=40mm]{img/grad.jpg}
\includegraphics[width=40mm]{img/gradFattal.jpg}
\includegraphics[width=40mm]{img/gradDe.png}
\caption{Poređenje rezultata dehazing-a iz rada [5] (''Fattal'') i naše implementacije koja koristi vođeni filter. a) originalna slika b) ''Fattal'' c) vođeni filter}
\label{grad}
\end{figure} 

Vođeni filter nije najbolje rešenje, ako posmatramo vizuelno rezultat. Međutim zbog složenosti vođenog filtera koja je $O(n)$, ovo je najbrža implementacija, koja daje dobre rezultate. U praktičnoj primeni brzina može da bude značajan faktor, dok vizuelna satisfakcija nije bitna ako sa slike mogu da se izvuku bitne informacije. Vođeni filter daje dobre rezultate prilikom otkrivanja detalja koji su bili zamagljeni, što je obično cilj praktičnih primena.

\section{Zaključak}%%%%%%%%%%%%%%%%%%%%%%%%%%%%%%%%%%%%%%

U ovom radu je dat pregled osnovnih filtara. Predstavljeni su rezultati filtriranja i osobine koje oni iskazuju prilikom podešavanja parametara filtriranja. Možemo da zaključimo da filteri za uglačavanje slike kao rezultat daju sliku čiji će pikseli imati srednju vrednost piksela iz okoline definisane prozorom filtriranja. Videli smo kako promena veličine prozora utiče na rezultat filtriranja i možemo da zaključimo da se povećanjem prozora dobija slika koja je više zamućena i time se narušavaju detalji slike, ali se postiže i bolje uklanjanje šuma. Predstavljen je i medijan filter kod koga se vrednost piksela rezultujuće slike računa kao vrednost koja je u sredini u odnosu na sve ostale. Ovaj filter pokazuje bolje rezultate ali je složeniji za implementaciju. Možemo da zaključimo da filteri za izvlačenje ivica služe za određivanje detalja. Pokazano je kako promenom tipa filtera (odnosno vrednosti prozora koji se koristi za filtriranje), dobijaju različiti rezultati i možemo da utvrdimo da korišćenjem drugog izvoda dobijamo finije ivice, nego korišćenjem prvog.

U radu je prikazan vođeni filter. Ovaj filter ima osobinu da prilikom uglačavanja slike vodi računa o ivicama, tj. ne narušava ivice. U odnosu na bilateralni filter, koji ima istu namenu, vođeni filter se pokazao kao bolji jer nema artifakte oko ivica na slikama. Takođe možemo da utvrdimo da je glavna prednost vođenog filtera, u odnosu na filtere koji imaju istu namenu, u tome što mu složenost ne zavisi od veličine prozora koji se koristi za filtriranje. 

Ovde je predstavljen vođeni filter, njegova definicija i osobine. Predstavljene su metode za izračunavanje vrednosti piksela koje ne zavise od veličine prozora. Na osnovu toga je napravljena referentna implementacija u programskom jeziku ''matlab'', koja je iskorišćena za analizu vođenog filtera. U radu su prikazane dobre strane ovog filtera, koje se ogledaju u tome da čuva ivice, da nema loših artifakta u blizini ivica. Predložena je metoda za uklanjanje izmaglice sa slika korišćenjem vođenog filtera, za filtriranje grube procene količine izmaglice na slici. Ovde smo mogli da uvidimo i loše strane vođenog filtera, kao što je pojava ''halo'' efekta u blizini ivica. 

Vođeni filter je upoređen sa drugim filterima, i možemo da zaključimo da za uglačavanje slike i uklanjanje šuma pokazuje vizuelno bolje rezultate u odnosu na njih. Što se tiče uklanjanja izmaglice možemo da zaključimo da se vođeni filter pokazao kao dobro rešenje koje daje zadovoljavajuće rezultate u odnosu na implementacije koje imaju veću složenost. U radu je pokazano da implementacija za uklanjanje izmaglice, koja koristi vođeni filter, pažljivim odabirom parametara daje dobre rezultate u odnosu na druge implementacije.

Možemo da zaključimo da vođeni filter može da se koristi u svim procesima obrade slike koji zahtevaju uglačavanje slike, a da se pri tome sačuvaju detalji slike. On ima svoja ograničenja kada su u pitanju slike kod kojih ivice nisu izražene (mala je razlika između piksela sa jedne i druge strane ivice ili prelaza). Sa druge strane omogućava implementaciju kod koje izračunavanje vrednosti piksela rezultujuće slike ne zavisi od veličine prozora koji se koristi za izračunavanje.

\newpage
\addcontentsline{toc}{section}{Literatura}
\begin{thebibliography}{10}

\bibitem{DigitalImageProcessing}
Rafael C. Gonzalez, Richard E. Woods,
\href{https://books.google.rs/books?id=8uGOnjRGEzoC&redir_esc=y}{''Digital Image Processing''},
Prentice Hall,
2008.

\bibitem{GuidedFilter1}
Kaiming He, Jian Sun, Xiaoou Tang,
\href{http://kaiminghe.com/publications/eccv10guidedfilter.pdf}{''Guided Image Filtering''},
European Conference on Computer Vision, pp 1-14,
2010.

\bibitem{GuidedFilter2}
Kaiming He, Jian Sun, Xiaoou Tang,
\href{http://kaiminghe.com/publications/pami12guidedfilter.pdf}{''Guided Image Filtering''},
IEEE Transactions on pattern analisys and machine intelligence, pp 1397-1409
2013.

\bibitem{GuidedFilter3}
Kaiming He, Jian Sun
\href{https://arxiv.org/pdf/1505.00996.pdf}{''Fast Guided Filter''},
arXiv.org at Cornell University Library,
2015.

\bibitem{Dehaze}
Raanan Fattal,
\href{http://citeseerx.ist.psu.edu/viewdoc/download?doi=10.1.1.456.2558&rep=rep1&type=pdf}{''Single image dehazing''},
Special Interest Group on Computer Graphics and Interactive techniques, pp 1-9,
2008.

\bibitem{Dehaze2}
Kaiming He, Jian Sun, Xiaoou Tang,
\href{http://citeseerx.ist.psu.edu/viewdoc/download?doi=10.1.1.456.2558&rep=rep1&type=pdf}{''Single Image Haze Removal
Using Dark Channel Prior''},
IEEE Transactions on pattern analisys and machine intelligence, pp 2341-2353,
2011.

\end{thebibliography}

\end{document}